% Encoding: utf-8

\chapterdoc{Example of the frustrated total internal reflection}
\chapterauthor{Olivier Castany, Céline Molinaro}

Frustrated total internal reflection is used as a validation example. 
Analytical and numerical results are compared.

\section{Presentation}

We consider the situation on figure~\ref{fig:FTIR}, where two half-spaces with indices $n_f$ and $n_b$ are separated by a medium of index $n_s$ and thickness $d$.
The three media are assumed to be lossless.
The incoming plane wave defines vector $k_x$ throughout the structure.
The reduced wave vector is $K_x=n_f \sin(\phi_i)$.
The scalar Helmholtz equation holds separatley inside the three media, 
$(\Delta+k_0^2 \varepsilon)\{E,H\}=0$,
and implies that there are at most two waves in each medium, with wave vector $\pm k_z$, given by $k_z^2 = k_0^2 n^2 - k_x^2$.

\begin{figure}[!h]
\includegraphics[width=\columnwidth]{fig/FrustratedTIR}
\caption{\label{fig:FTIR}Frustrated total internal reflection with input and output plane waves. }
\end{figure}

We consider the separation medium.
Since $k_x$ is real, $k_z$ is either purely real or purely complex.
The first case happens for incidence angles $\phi_i$ smaller than the critical angle $\phi_{ic}$, given by $\sin(\phi_{ic}) = n_s/n_f$.
In this case, two plane waves are superimposed.
The second case corresponds to the total internal reflexion in the front half-space medium. 
There is only an evanescent wave in the separation medium.
However, if the third material is close enough, a plane wave is transmitted from the evanescent wave.
This phenomenon is the frustrated total internal reflexion.
In this case, there are two evanescent waves superimposed in the separation medium.

Due to the ($xz$) mirror symmetry, $s$ and $p$ modes can be considered separately. 
For $s$ polarization, we have
$$
\vec E = E(x,z)\ \vec y
\quad\mathrm{and}\quad
\vec H = \frac{1}{i k_0} 
\begin{pmatrix}
-\partial_z E \\
\phantom{-} 0 \\
\phantom{-} \partial_x E
\end{pmatrix}.
$$
For $p$ polarization, we have
$$
\vec H = H(x,z)\ \vec y
\quad\mathrm{and}\quad
\vec E = \frac{i}{k_0 \varepsilon} 
\begin{pmatrix}
-\partial_z H \\
\phantom{-} 0 \\
\phantom{-} \partial_x H
\end{pmatrix}.
$$

In the next sections, we will study the anatomy of the waves in detail. 
For that purpose, the Poynting vector gives useful physical insight.
In Gaussian units, the time average of the Poynting vector is
$$\langle \vec{\Pi} \rangle_t  =  
\frac{c}{8\pi} \Re\left(\vec{E}\times \vec{H}^*\right).$$

\section{Anatomy of a single wave}

If we consider an $s$-polarized wave defined by 
$$
\vec E = E(x,y)\ \vec y
\quad\mathrm{with}\quad
E(x,y) = E_0\ e^{-i\omega t + i(k_x x + k_z z)},$$
we deduce
$\quad
\vec H = E(x,y)
\begin{pmatrix}
- K_z\\
\phantom{-} 0\\
\phantom{-} K_x
\end{pmatrix},
$
$$
\vec H^* = E_0^*\ e^{i\omega t - i(k_x x + k_z^* z)}
\begin{pmatrix}
- K_z^*\\
\phantom{-} 0\\
\phantom{-} K_x
\end{pmatrix},
$$
$$
\mathrm{and}\quad
\langle \vec{\Pi} \rangle_t  = \frac{c}{8\pi} 
|E_0|^2 e^{-2 k_z'' z}
\begin{pmatrix}
K_x\\
0\\
K_z'
\end{pmatrix}.
$$
If we consider a $p$-polarized wave defined by 
$$
\vec H = H(x,y)\ \vec y
\quad\mathrm{with}\quad
H(x,y) = H_0\ e^{-i\omega t + i(k_x x + k_z z)},$$
we deduce
$\quad
\vec E = \displaystyle \frac{H(x,y)}{\varepsilon}
\begin{pmatrix}
\phantom{-} K_z\\
\phantom{-} 0\\
- K_x
\end{pmatrix},
$
\begin{equation*}
\vec H^* = H_0^*\ e^{i\omega t - i(k_x x + k_z^* z)}\ \vec y,
\end{equation*}
$$
\mathrm{and}\quad
\langle \vec{\Pi} \rangle_t  = \frac{c}{8\pi} 
\frac{|H_0|^2}{\varepsilon} e^{-2 k_z'' z}
\begin{pmatrix}
K_x\\
0\\
K_z'
\end{pmatrix}.
$$

In these expressions, we defined $k_z = k_z' + i k_z''$.
We observe that the Poynting vector is parallel to the real part of the wave vector.
If $k_z$ is real, the wave amplitude is constant along $z$.
If $k_z$ is purely complex, there is no energy flow in the $z$ direction, and the wave decays exponentially in the $z$ direction.
Figure~\ref{fig:Plane_wave_real_complex} represents the two cases.

\begin{figure}[h!]
\includegraphics[width=\columnwidth]{fig/Plane_wave_real_complex}
\caption{\label{fig:Plane_wave_real_complex}Real part of the wave vector. The thickness of the line indicates the intensity of the wave.}
\end{figure}


\section{Reflection on an interface, fields and Poynting vector}

We consider an incident wave partially reflected at $z=0$ by a structure that does not change the type $s$ or $p$ of the polarization.
The waves may either be plane waves or evanescent waves.
The incident wave is named with a ``$+$'' subscript and the reflected wave with a ``$-$'' subscript.
The complex reflection coefficients for each mode are called $r_s$ and $r_p$.
%
The total field is $\vec E = \vec E^+ + \vec E^-$
and for the $s$ polarisation, we consider
$$
\left\{
\begin{array}{l}
\vec E^+ = E^+\ \vec y \\
\vec E^- = E^-\ \vec y
\end{array}
\right.
\quad\mathrm{with}\quad
\left\{
\begin{array}{l}
E^+ = E_0^+ e^{-i\omega t + i(k_x x + k_z z)}  \\
E^- = E_0^- e^{-i\omega t + i(k_x x - k_z z)}.
\end{array}
\right.
$$
The amplitudes of the incident and reflected waves are connected by $E_0^- = r_s E_0^+$.
The magnetic excitation is 
$$
\vec H = \vec H^+ + \vec H^- = 
E^+ 
\begin{pmatrix}
- K_z \\
\phantom{-} 0 \\
\phantom{-} K_x 
\end{pmatrix}
+
E^-
\begin{pmatrix}
K_z \\
0 \\
K_x 
\end{pmatrix}.
$$
The Poynting vector is
\begin{align*}
\langle \vec{\Pi} \rangle_t = 
\frac{c}{8\pi} |E_0^+|^2 \left\{
e^{-2 k_z'' z} 
\begin{pmatrix}
K_x \\
0 \\
K_z'
\end{pmatrix}
+ 
|r_s|^2
e^{2 k_z'' z}
\begin{pmatrix}
\phantom{-} K_x \\
\phantom{-} 0 \\
-K_z'
\end{pmatrix}
+ \right.\\
\left. +
2\ |r_s|\ 
\begin{pmatrix}
K_x \cos(\theta_{s} - 2 k_z' z)\\
0 \\
K_z'' \sin(\theta_{s} - 2 k_z' z)
\end{pmatrix}
\right\},
\end{align*}
where we defined $r_s = |r_s|\ e^{i\theta_{s}}$.
The first and second terms correspond to the incident and reflected waves.
The third term arises from the interference of the two waves.
If $K_z$ is real, the expression in curly braces becomes
$$
\begin{pmatrix}
K_x \\
0 \\
K_z
\end{pmatrix}
+ 
|r_s|^2
\begin{pmatrix}
\phantom{-} K_x \\
\phantom{-} 0 \\
- K_z
\end{pmatrix}
+ 2\ |r_s|\ 
\begin{pmatrix}
K_x \cos(\theta_{s} - 2 k_z z)\\
0 \\
0 \\
\end{pmatrix}.
$$
If $K_z$ is purely complex, it becomes
$$
e^{-2 k_z'' z} 
\begin{pmatrix}
K_x \\
0 \\
0
\end{pmatrix}
+ 
|r_s|^2
e^{2 k_z'' z}
\begin{pmatrix}
K_x \\
0 \\
0
\end{pmatrix}
+ 2\ 
\begin{pmatrix}
K_x r_s'\\
0 \\
K_z'' r_s''
\end{pmatrix},
$$
which exhibits an energy flow in the $z$ direction, proportionnal to $K_z'' r_s''$.

For $p$ polarisation, we consider 
$$
\left\{
\begin{array}{l}
\vec H^+ = H^+\ \vec y \\
\vec H^- = H^-\ \vec y
\end{array}
\right.
\quad\mathrm{with}\quad
\left\{
\begin{array}{l}
H^+ = H_0^+ e^{-i\omega t + i(k_x x + k_z z)}  \\
H^- = H_0^- e^{-i\omega t + i(k_x x - k_z z)}.
\end{array}
\right.
$$
The amplitudes of the incident and reflected waves are connected by $H_0^- = r_p H_0^+$.
The electric field is
$$
\vec E = \vec E^+ + \vec E^- = 
\frac{H^+}{\varepsilon}
\begin{pmatrix}
\phantom{-} K_z \\
\phantom{-} 0 \\
- K_x 
\end{pmatrix}
+
\frac{H^-}{\varepsilon}
\begin{pmatrix}
- K_z \\
\phantom{-} 0 \\
- K_x 
\end{pmatrix}.
$$
The Poynting vector is the same expression as for the $s$ polarization, when $r_s$ is replaced by $r_p = |r_p|\ e^{i\theta_{p}}$ and $|E_0^+|^2$ is replaced by $|H_0^+|^2/\varepsilon$,
\begin{align*}
\langle \vec{\Pi} \rangle_t = 
\frac{c}{8\pi} \frac{|H_0^+|^2}{\varepsilon} \left\{
e^{-2 k_z'' z} 
\begin{pmatrix}
K_x \\
0 \\
K_z'
\end{pmatrix}
+ 
|r_p|^2
e^{2 k_z'' z}
\begin{pmatrix}
\phantom{-} K_x \\
\phantom{-} 0 \\
-K_z'
\end{pmatrix}
+ \right.\\
\left. +
2\ |r_p|\ 
\begin{pmatrix}
K_x \cos(\theta_{p} - 2 k_z' z)\\
0 \\
K_z'' \sin(\theta_{p} - 2 k_z' z)
\end{pmatrix}
\right\}.
\end{align*}
The analysis is identical to the $s$ polarization. 
If $K_z$ is purely complex, there is an energy flow in the $z$ direction, proportionnal to $K_z'' r_p''$ (see figure~\ref{fig:Wave_evanescent_poynting}).

\begin{figure}[!h]
\includegraphics[width=\columnwidth]{fig/Wave_evanescent_poynting}
\caption{\label{fig:Wave_evanescent_poynting}Poynting vector in the case of evanescents waves ($k_z$ purely complex).}
\end{figure}

\section{Expression of the reflexion coefficients for an interface between two media}

We consider the reflexion on the interface between the separation medium and the back half-space.
For comparison with well known formulas, we will name these media ``1'' and ``2'' respectively.
The reflection and transmission coefficients for the $s$ polarization are\cite{Wikipedia_Fresnel}
$$
r_s = \frac{k_{z1}-k_{z2}}{k_{z1}+k_{z2}} 
\quad\textrm{and}\quad
t_s = 1 + r_s = \frac{2\ k_{z1}}{k_{z1}+k_{z2}}.
$$
The expressions are valid for complex wave vectors and we deduce
$$
r_s' = \frac{|k_{z1}|^2 - |k_{z2}|^2}{|k_{z1} + k_{z2}|^2}
\quad\textrm{and}\quad
r_s'' = \frac{2 (k_{z1}'' k_{z2}' - k_{z1}' k_{z2}'')}{|k_{z1} + k_{z2}|^2}
$$
In the case of an evanescent wave in the separation medium and a plane wave in the back half-space, we have $k_{z1} = i k_{z1}''$ and $k_{z2} = k_{z2}'$, which implies
$$
r_s' = \frac{k_{z1}''^2 - k_{z2}^2}{k_{z1}''^2 + k_{z2}^2}
\quad\textrm{and}\quad
r_s'' = \frac{2 k_{z1}'' k_{z2}}{k_{z1}''^2 + k_{z2}^2}.
$$
The last expression show that $r_s''>0$, the energy flow is directed to the right, as expected intuitively.

For the $p$ polarization, we have\cite{Wikipedia_Fresnel}
$$
r_p=\frac{\epsilon_2k_{z1}-\epsilon_1k_{z2}}{\epsilon_2k_{z1}+\epsilon_1k_{z2}}
\quad\textrm{and}\quad
t_p = \frac{2\epsilon_2 k_{z1}}{\epsilon_2k_{z1}+\epsilon_1k_{z2}},
$$
\begin{flalign*}
\textrm{from which we deduce } 
r_p' = \frac{\epsilon_2 |k_{z1}|^2 - \epsilon_1 |k_{z2}|^2}{|\epsilon_2 k_{z1} + \epsilon_1 k_{z2}|^2} && \\
\textrm{and}\quad
r_p'' = \frac{2 \epsilon_1\epsilon_2(k_{z1}'' k_{z2}' - k_{z1}' k_{z2}'')}{|\epsilon_2 k_{z1} + \epsilon_1 k_{z2}|^2}. &&
\end{flalign*}
In the case of an evanescent wave in the separation medium and a plane wave in the back half-space, we have 
$$
r_p' = \frac{\epsilon_2 k_{z1}''^2 - \epsilon_1 k_{z2}^2}{\epsilon_2 k_{z1}''^2 + \epsilon_1 k_{z2}^2}
\quad\textrm{and}\quad
r_p'' =  \frac{2 \epsilon_1\epsilon_2 k_{z1}'' k_{z2}'}{\epsilon_2 k_{z1}''^2 + \epsilon_1 k_{z2}^2}
.
$$
The last expression show that $r_p''>0$, the energy flow is directed to the right, as expected intuitively.



\section{Frustrated internal total reflection}

For $s$ polarization, we consider the waves $E_i$ and $E_r$ in the incident medium, $E^+$ and $E^-$ in the separation medium, and $E_t$ in the back half-space.
Continuity of the field at $z=0$ implies


\subsection{Electric fields and wave vectors}
To study the frustrated total internal reflection, calculations are done in the case where the angle of incidence is larger than the critical angle :
$\Phi_i > \Phi_{ic}$
Five waves are studied. Each wave has the same wave vector $k_x$.
\begin{center}
$k_x=n_fk_0\sin(\Phi_i)$ and $k_x=n_bk_0\sin(\Phi_t)$\\
\end{center}

The first two are in the front medium where the wave vector along $z$, $k_{fz}\in \mathbb{R}$.
\begin{itemize}
\item  Incident wave  $\vec{E}_i(x,z)=\vec{E}_i(z)e^{-i\omega t+i(k_{fz}z+k_xx)}$
\item Reflected wave  $\vec{E}_r(x,z)=\vec{E}_r(z)e^{-i\omega t+i(-k_{fz}z+k_xx)}$\\
where $k_{fz}=n_fk_0\cos(\Phi _i)$.
\end{itemize}
Two others are in the air where $k_{nz}\in i\mathbb{R}$. One is the transmitted wave from the front medium and the other is the reflected wave.
\begin{itemize}
\item Incident wave 
$\vec{E}_{ni}(x,z)=\vec{E}_ni(z)e^{-i\omega t+i(k_{nz}z+k_xx)}$
\item Reflected wave 
$\vec{E}_{nr}(x,z)=\vec{E}_nr(z)e^{-i\omega t+i(-k_{nz}z+k_xx)}$\\
where $k_{nz}=ik_0\sqrt{n_f^2\sin^2(\Phi _i)-n^2}$.
\end{itemize}
The last wave is in the back medium where $k_{b}\in \mathbb{R}$.
\begin{itemize}
\item Transmitted wave
$\vec{E}_t(x,z)=\vec{E}_t(z)e^{-i\omega t+i(k_{bz}z+k_xx)}$\\
where $k_{bz}=n_bk_0\cos(\Phi _t)$.
\end{itemize}
In this medium the plane wave has an angle \\
$$
\Phi_t=\arcsin (\frac{n_{fz}\sin(\Phi_i)}{n_b})
$$

\subsection{Continuity equations} 
The electric waves follow the $y$ axe.\\
Continuity equations $z=0$
\begin{equation}
\left\lbrace
\begin{array}{ccc}\label{eq:continuity0}
E_{ni}&=r_{fn}E_{nr}+t_{nf}E_i\\
E_r&=r_{nf}E_{i}+t_{fn}E_{nr}
\end{array}\right.
\end{equation}
Continuity equations $z=z_b$
\begin{equation}
\left\lbrace
\begin{array}{ccc}\label{eq:continuityd}
E_t=t_{bn}E_{ni}\\
E_{nr}=r_{bn}E_{ni}
\end{array}\right.
\end{equation}
$E_{ni}$ and $E_{nr}$ can be written at $z=0$ or $z=z_b$. The relations between them are
\begin{equation}
\left\lbrace
\begin{array}{ccc}\label{eq:0tod}
E_{ni}(0)=E_{ni}(d)e^{-ik_{nz}d}\\
E_{nr}(0)=E_{nr}(d)e^{ik_{nz}d}
\end{array}\right.
\end{equation}

With the equations~\eqref{eq:continuity0},~\eqref{eq:continuityd} and~\eqref{eq:0tod}, we got\\
\begin{equation}
E_{ni}(0)=\frac{t_{nf}}{1+r_{nf}r_{bn}e^{i2k_{nz}d}}E_i(0)
\end{equation}

\begin{equation}
E_{nr}(0)=\frac{t_{nf}t_{bn}e^{i2k_{nz}d}}{1+r_{nf}r_{bn}e^{i2k_{nz}d}}E_i(0)
\end{equation}

\begin{equation}\label{eq:reflectivewave}
E_{r}(0)=\frac{r_{fn}+r_{bn}e^{i2k_{nz}d}}{1+r_{nf}r_{bn}e^{i2k_{nz}d}}E_i(0)
\end{equation}

\begin{equation}\label{eq:transmittedwave}
E_{t}(z_b)=\frac{t_{nf}t_{bn}e^{ik_{nz}d}}{1+r_{nf}r_{bn}e^{i2k_{nz}d}}E_i(0)
\end{equation}
According to equation~\eqref{eq:reflectivewave}, the amplitude coefficient reflection for the interface is\\
$$
r=\frac{r_{fn}+r_{bn}e^{i2k_{nz}d}}{1+r_{nf}r_{bn}e^{i2k_{nz}d}}
$$
The reflection coefficient is given by
\begin{equation*}
R=\mid r\mid^2
\end{equation*}
According to equation~\eqref{eq:transmittedwave}, the amplitude coefficient transmission for the interface is\\
$$
t=\frac{t_{nf}t_{bn}e^{ik_{nz}d}}{1+r_{nf}r_{bn}e^{i2k_{nz}d}}
$$
The transmission coefficient is given by
\begin{equation*}
T=\frac{n_b\cos\Phi _t}{n_f\cos\Phi _i}\mid t\mid^2
\end{equation*}
The conservation of energy $R+T=1$ is respected in the cases of polarisation p and polarisation s.


\subsection{Front medium}
In this medium, there are two waves, the incident and the reflected. The total electric field is:
$\vec{E}_f = \vec{E}_i+\vec{E}_r$\\
On the $y$ axe:
\begin{eqnarray*}
E_f = E_i(0)\ 
\big\{&  e^{-i\omega t+i(k_{fz}z+k_xx)} + \cdots  \\
 &+ r_{tot} \ e^{-i\omega t+i(-k_{fz}z+k_xx)} \big\}
\end{eqnarray*}
with 
\begin{eqnarray*}
r_{tot} & = & \frac{r_{fn}+r_{bn}e^{i2k_{nz}d}}{1+r_{nf}r_{bn}e^{i2k_{nz}d}}\\
& = & \Re(r_{tot})+i\Im(r_{tot})
\end{eqnarray*}
%The conugate magnetic field is:\\
%\begin{eqnarray*}
%\vec{H}^*=\frac{E_i(0)^*}{\omega \mu_0}\{&k_{fz}(-e^{i\omega t-i(k_{fz}z+k_xx)}\\
%+r_{tot}^*e^{i\omega t+i(k_{fz}z-k_xx)})\vec{x}\\
%+&k_x(e^{i\omega t-i(k_{fz}z+k_xx)}\\
%+r_{tot}^*e^{i\omega t+i(k_{fz}z-k_xx)})\vec{z}\}
%\end{eqnarray*}
The Poynting vector:\\
\begin{align*}
\langle \vec{\Pi} \rangle _t =\displaystyle\frac{\displaystyle\mid E_i(0)\mid ^2}{\displaystyle2\omega \mu_0}\Big\{&
\begin{pmatrix}
k_x\\
k_{fz}
\end{pmatrix}
+\mid r_{tot}\mid^2
\begin{pmatrix}
k_x\\
-k_{fz}
\end{pmatrix}\\
&+2
\begin{pmatrix}
\Re(r_{tot})\cos(2k_{fz}z\\
0
\end{pmatrix} 
\Big\}
\end{align*}

\subsection{Air medium}
In this medium, there are two waves. The total electric field is:
$\vec{E}_n=\vec{E}_{ni}+\vec{E}_{nr}$\\
On the $y$ axe:\\
\begin{eqnarray*}
E_n = E_{ni}(0)\ 
\big\{&  e^{-i\omega t+i(k_{nz}z+k_xx)} + \cdots  \\
 &+ r_n \ e^{-i\omega t+i(-k_{nz}z+k_xx)} \big\}
\end{eqnarray*}
with:
\begin{eqnarray*}
r_n & = & r_{bn}e^{i2k_{nz}d}\\
& = & \Re(r_n)+i\Im(r_n)
\end{eqnarray*}

%The conjugate magnetic fiels has been calculated in the most general case, with $k_{nz}\in \mathbb{C}$:\\
%\begin{eqnarray*}
%\vec{H}^*=\frac{E_{ni}(0)^*}{\omega \mu_0}\{
%&k_{nz}^*\big(-e^{i\omega t-i(k_{nz}^*z+k_xx)}\\
%+r_n^*e^{i\omega t+i(k_{nz}^*z-k_xx)}\big)\vec{x}\\
%+&k_x\big(e^{i\omega t-i(k_{nz}^*z+k_xx)}\\
%+e^{i\omega t+i(k_{nz}^*z-k_xx)}\big)\vec{z}\}
%\end{eqnarray*}
%We write:\\
%$$
%k_{nz}=k_{nz}'+ik_{nz}''
%$$
%The Poynting vector is:\\
%$
%\langle \vec{\Pi} \rangle _t =\displaystyle\frac{\displaystyle\mid E_{ni}(0)\mid ^2}{\displaystyle2\omega \mu_0}\Big\{
%e^{-2k_{nz}''z}
%\begin{pmatrix}
%k_x\\
%k_{nz}'
%\end{pmatrix}\\
%+\mid r_n\mid ^2e^{2k_{nz}''z}
%\begin{pmatrix}
%k_x\\
%-k_{nz}'
%\end{pmatrix}\\
%+2\mid r_n\mid
%\begin{pmatrix}
%k_x\cos(\theta _n-2k_{nz}'z)\\
%k_{nz}''\sin(\theta _n -2k_{nz}'z)
%\end{pmatrix}
%\Big\}
%$\\
In the case of the frustrated total internal reflexion, the wave in the air medium is an evanescent wave, $k_{nz} \in i\mathbb{R}$. The Poynting vector is:\\
$
\langle \vec{\Pi} \rangle _t =\displaystyle\frac{\displaystyle\mid E_{ni}(0)\mid ^2}{\displaystyle2\omega \mu_0}\Big\{
e^{2k_{nz}z}
\begin{pmatrix}
k_x\\
0
\end{pmatrix}
+\mid r_n\mid ^2e^{-2k_{nz}z}
\begin{pmatrix}
k_x\\
0
\end{pmatrix}\\
+2
\begin{pmatrix}
k_x\Re(r_n)\\
ik_{nz}^*\Im(r_n)
\end{pmatrix}
\Big\}
$\\
The third term is an interference terms. Along $z$ we get:\\
$$
ik_{nz}^*\sin(\theta _n) =\frac{ k_0^3n_b\cos(\Phi _t)(n_f^2\sin ^2(\Phi _i)-n^2)e^{ik_{nz}d}}{\mid k_{nz}+k_{fz}\mid ^2}
$$
This expression is positive so the interference term is oriented to the positive $z$.
\subsection{Back medium}
In this medium, there is one wave, the transmitted wave.With equation~\eqref{eq:transmittedwave}, on the $y$ axe:\\
\begin{equation*}
E_t = E_i(0)te^{-i\omega t+i(k_bz+k_xx)}
\end{equation*}
with:
$$
t = \frac{t_{nf}t_{bn}e^{ik_{nz}d}}{1+r_{nf}r_{bn}e^{i2k_{nz}d}}
$$
%The conjugate magnetic field is:\\
%\begin{eqnarray*}
%\vec{H}^*= \frac{E_i(0)^*}{\omega \mu_0}\Big\{-k_t^*be^{i\omega t-i(k_bz+k_xx)}\vec{x}\\
%+k_xt^*e^{i\omega t-i(k_bz+k_xx)}\vec{z}\Big\}
%\end{eqnarray*}
The Poynting vector is:\\
$$
\langle \vec{\Pi} \rangle _t = \frac{\mid E_i(0)\mid ^2\mid t\mid ^2}{2\omega \mu_0}
\begin{pmatrix}
k_x\\
k_b
\end{pmatrix}
$$
