% Encoding: utf-8

\chapterdoc{Various examples}
\chapterauthor{Céline Molinaro, Olivier Castany}

\section{Glass layer}

In \verb/Interferences.py/, we consider the glass plate from figure~\ref{fig:Glass-layer}, where light arrives with a 30° incidence angle ($K_x=0.5$).
The reflection and transmission coefficients $R$ and $T$ are plotted as a function of the glass thickness, for both polarizations and interferences are visible.

\begin{figure}[!h]
\includegraphics[width=\columnwidth]{fig/Glass-layer}
\caption{\label{fig:Glass-layer}Glass plate with multiple reflections inside.}
\end{figure}

\section{Reflection on an interface}
A plane wave is reflected and refracted by an interface, as represented on figure~\ref{fig:Reflection-interface}. 

\begin{figure}[!h]
\includegraphics[width=\columnwidth]{fig/Reflection-interface}
\caption{\label{fig:Reflection-interface}Reflection of a plane wave on an interface between two materials.}
\end{figure}

In \verb/interface-Jones.py/, we consider an interface with $n_1=1.0$ and $n_2=1.5$.
The Jones matrix for reflection and transmission are calculated, with linear or circular bases.

In \verb/ interface-reflection.py/, the power reflection and transmission coefficients $R$ and $T$ are plotted for both polarizations as a function of the reduced wave vector $K_x$. 
We observe that $R+T=1$.
The coefficient $|t^2|$ is also plotted and is clearly different from $T$.
The result from Berreman4x4 and the analytic solution are plotted together and are identical.
The refractive indices can be given from the command line and total internal reflection is observed if $n_1>n_2$.


\section{Bragg Mirror}
We consider TiO$_2$/SiO$_2$ Bragg mirrors as presented on figure~\ref{fig:Bragg}.
In \verb/validation-Bragg.py/, the reflection coefficient $R$ is calculated with Berreman4x4 and compared to the analytical result, for different incidence angles and polarizations. 
The results are identical.
In \verb/Bragg-example.py/, the reflection and transmission coefficients $R$ and $T$ are calculated for a Bragg mirror with $8.5$ periods at normal incidence.
\begin{figure}[!h]
\includegraphics[width=\columnwidth]{fig/Bragg}
\caption{\label{fig:Bragg}TiO$_2$/SiO$_2$ Bragg mirror with two periods.}
\end{figure}

\section{Twisted nematic liquid crystal}
We consider a twisted nematic liquid crystal between two glass substrates as represented on figure~\ref{fig:nematic}.  
In \verb/twisted-nematic.py/, the transmission coefficient $T$ calculated with Berreman4x4 is compared with the Gooch-Tarry law.
Agreement is excellent when the twisted layer is divided into 18 divisions.
When only 7 divisions are used, there is a slight discrepancy in the high wavenumber range.

\begin{figure}[!h]
\begin{center}
\includegraphics{fig/nematic}
\caption{\label{fig:nematic} Twisted nematic liquid crystal between glass substrates with polarizer and analyzer.}
\end{center}
\end{figure}

\section{Cholesteric liquid crystal}
\begin{figure}[!h]
\begin{center}
\includegraphics{fig/cholesteric}
\end{center}
\caption{\label{fig:cholesteric}A cholesteric liquid crystal where p is the pitch}
\end{figure}
The characteristic of a cholesteric liquid crystal as shown on figure~\ref{fig:cholesteric} is its helical structure. Its director vector is rotated. The pitch represents the period of the rotation variation.
This example deals with a $n$ layers cholesteric liquid crystal. In \verb/validation-cholesteric.py/, the calculated and theoritical reflection spectra of the $n$ cholesteric layers are drawn. The theoritical expression used to draw a theoritical reflection power coefficient comes from \cite{Yang} and \cite{Chandrasekhar}. See source code for further details. In \verb/cholesteric.py/, transmission (for different polarization : right-circular, $p$, unpolarized...) and reflection spectra are drawn. Eigenvalues and eigenvectors of the transmission matrix for different cases are calculated. See source code for further details. 

