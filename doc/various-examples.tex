% Encoding: utf-8

\chapterdoc{Various examples}
\chapterauthor{Olivier Castany, Céline Molinaro}

\section{Reflection on an interface}
This is the  simplest case as shown on figure ~\ref{fig:Reflection-interface}. A wave is reflected and refracted by an interface. This interface is characterized by the  transmission and reflection power coefficient. See source code for further details.
\begin{figure}[!h]
\includegraphics[width=\columnwidth]{fig/Reflection-interface}
\caption{\label{fig:Reflection-interface}Input and output plane waves at the interface.}
\end{figure}


\section{Bragg Mirror}
Figure ~\ref{fig:Bragg} represents a Bragg mirror with two layers. Each layers is composed of two thin dielectric layers : SiO$_2$ and TiO$_2$. An incident wave, its reflecteds waves and its transmitted wave are represented. Bragg mirrors are used to produce ultra-high relfectivity. This example is used to draw the global reflection power coefficient for all layers depending on the wave length. See source code for further details.
\begin{figure}[!h]
\includegraphics[width=\columnwidth]{fig/Bragg}
\caption{\label{fig:Bragg}Bragg Mirror with two layers}
\end{figure}

\section{Cholesteric liquid crystal}
The structure of a cholesteric crystal is helical. That means the director vector is rotated as shown on figure ~\ref{fig:cholesteric}. The pitch is the varitation period of this rotation.
This example deals with $n$ cholesteric liquid crystal layers. The reflection spectra can be drawn. The theoritical expression comes from "Fundamentals Crystal Devices" from D.Yang and S.Wu. See source code for further details.
\begin{figure}[!h]
\includegraphics[width=\columnwidth]{fig/cholesteric}
\caption{\label{fig:cholesteric}Director vector of a cholesteric liquid crystal where p is the pitch}
\end{figure}
