% Encoding: utf-8

\chapterdoc{Validation examples taken from Fujiwara's book}
\chapterauthor{Olivier Castany}

We reproduce the situation presented by Fujiwara in his book \emph{Spectroscopic Ellipsometry} \cite{Fujiwara}, section~6.4.2, p.~241--243 and section~6.4.1, p.~237--239.
He presents detailed calculations and intermediate steps that are useful for testing our code.


\section{Example of section 6.4.2}

The situation is depicted on figure~\ref{fig:situation642}.
A uniaxial film is formed on a silicon substrate.
The incident medium is air and the silicon substrate has refractive index $n_t = 3.898 + 0.016i$.
The thickness of the film is $d = 100$~nm and the refractive indices are $n_o = 2.0$ and $n_e=2.5$. 
The orientation of the film is given by Euler angles $\phi_E=\pi/4$ and $\theta_E = \pi/4$.
Angle $\phi_E$ is the first Euler angle, inducing a rotation of axis $x$ around axis $z$, leading to axis $x'$.
Angle $\theta_E$ is the seconde Euler angle, inducing a rotation of the axis of the material around $x'$.

\begin{figure}[b]
\includegraphics[width=\columnwidth]{fig/structure-Fujiwara-642}
\caption{\label{fig:situation642}Geometry of the situation treated by Fujiwara in his book, section~6.4.2.}
\end{figure}

In \verb/validation-Fujiwara-642.py/, we reproduce Fujiwara's results with our code and obtain the same values, except for a few places where a sign is reversed.
The reason is that Fujiwara uses the convention of ellipsometry for orienting $E_{rp}$, which is the opposite of our convention.
We also reproduce figure~6.19, p.~242, when the orientation of the anisotropic film is varied.

% Ugly trick to improve column placement...
\vspace{10cm}
\mbox{}
\vspace{7cm}

\section{Example of section 6.4.1}

\begin{figure}[b]
\includegraphics[width=\columnwidth]{fig/structure-Fujiwara-641}
\caption{\label{fig:situation641}Geometry of the situation treated by Fujiwara in his book, section~6.4.1.}
\end{figure}

In \verb/validation-Fujiwara-641.py/, light is reflected in the air by an anisotropic substrate as depicted on figure~\ref{fig:situation641}.
We reproduce figure~6.16, p.~238, when the angle of incidence $\phi_i$ is varied and figure~6.17, p.~239, when the Euler angle $\phi_E$ of the anisotropic substrate is varied.
The difference with section~6.4.2 is that the substrate is anisotropic and the code uses a general \verb/HalfSpace/ instead of an \verb/IsotropicHalfSpace/.



