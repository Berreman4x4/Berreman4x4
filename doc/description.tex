% Encoding: utf-8

\chapterdoc{Description of Berreman4x4}
\chapterauthor{Olivier Castany}

Electromagnetic wave propagation in stratified media is important in several applications like ellipsometry analysis, Bragg mirrors or twisted liquid crystal structures.
Propagation of plane waves in isotropic media can be solved with 2$\times$2 propagation matrix methods \cite{BornWolf}.
In the case of anisotropic layers, a 4$\times$4 propagation matrix method was developed \cite{1970_TeilerHenvis, Berreman} and is now known as Berreman's method. 

The present work is an open-source implementation of the method in Python.
This programming language was chosen for ease of use, readability of the source code, portability and availability of scientific libraries (NumPy and SciPy).
A drawback is slow speed, because the language is interpreted%
\footnote{%
High speed calculations are possible with NumPy and SciPy if array operations are vectorized.
However the structure of Berreman4x4 is difficult to vectorize with enough generality.}.

Calculations are based on articles from Berreman \cite{Berreman} (1972) and Schubert \cite{Schubert} (1996).
Application to ellipsometry is based Fujiwara's book \cite{Fujiwara} (2007).
General references for optics in anisotropic media are found in the books of Born and Wolf \cite{BornWolf} and Jackson \cite{Jackson}.

\section{Presentation}

As described on figure~\ref{fig:geometry}, we consider a stratified sample with layers invariant in the $(x,y)$ plane and stacked in the $z$ direction, starting from $z_f=0$.
Because of the translation invariance, modes with bounded fields can be classified according to wave numbers $(k_x,k_y)\in\mathbb{R}^2$.
Without loss of generality, we consider plane waves in the $(x,z)$ direction, i.e. $k_y=0$.

The front half-space is isotropic and a plane wave can be decomposed into $s$ and $p$ polarizations.
The $s$ polarization is a wave with electric field perpendicular (\emph{senkrecht}) to the plane of incidence, i.e. along $y$.
The $p$ polarization is a wave with electric field parallel to the plane of incidence.
A plane wave $i$ is incident from the front half-space with incidence angle $\phi_i$ and reflected into a plane wave $r$ with the same angle.
Angles are oriented by the $y$ direction.
In the general case, the back half-space is anisotropic and two transmitted plane waves $t$ are induced, with the same $k_x$, but different angles $\phi_t$.
The electric vector is not necessarily perpendicular to $\vec k$ and $s$ and $p$ waves can not be considered.

\begin{figure}
\includegraphics[width=\columnwidth]{fig/geometry}
\caption{\label{fig:geometry}Geometry of the sample, with input and output plane waves. }
\end{figure}

\section{Conventions and units}

Formulas are mostly based on Schubert's article \cite{Schubert}, with one difference being the orientation of the incident $p$ polarization.
Schubert takes the convention used in ellipsometry, where the base electric vectors for $p$ polarization point towards the sample in the front and exit half spaces%
\footnote{In Fujiwara's book, reference \onlinecite{Fujiwara}, p. 223, the base electric vectors for $p$ polarization point towards the $-z$ direction, which leads to the same definition as Schubert's.}.
In our work, we used a different convention, in which the base electric vectors point towards the $x$ direction in both half-spaces.
With this convention, the base electric vectors for the incident, reflected and transmitted waves are all in the same direction in the case of reflection with normal incidence.

Gaussian units are used in the detail of the calculation.
This makes it easier to compare our results with past literature, which uses Gaussian units for the most part.
Of course, reflection and transmission coefficients have no unit and do not depend on the choice of unit.
Conversion between units is presented in the appendix on electromagnetic units in reference~\onlinecite{Jackson}.
In Gaussian units, Maxwell's equations read
$$
\nabla\times E = -\frac{1}{c} \frac{\partial B}{\partial t} 
\quad \mathrm{and} \quad
\nabla\times H = \frac{1}{c} \frac{\partial D}{\partial t}.
$$
We consider linear materials with tensorial constitutive relations
$D = \varepsilon E$ and $B = \mu H$.
In Gaussian units, $\varepsilon_0=\mu_0=1$ and the relation to S.I. units is
$$
E_\mathrm{S.I.} = \frac{E}{\sqrt{4\pi\varepsilon_0}}
\quad \mathrm{and} \quad
H_\mathrm{S.I.} = \frac{H}{\sqrt{4\pi\mu_0}},
$$
where $\varepsilon_0$ and $\mu_0$ are the usual S.I. values.
%
The convention for time-varying complex fields is taken as $\exp(-i\omega t)$, and Maxwell's equations read
$$
\nabla\times E = i k_0 H
\quad \mathrm{and} \quad
\nabla\times H = -i k_0 \varepsilon E,
$$
with $k_0=\omega/c$.
We consider non-magnetic materials, i.e. $\mu = 1$.
The permittivity $\varepsilon$ may be complex. 
For example, isotropic materials with $\Im(\varepsilon)>0$ are lossy.

\section{Propagation inside the sample}

In the stratified sample, Maxwell's equations lead to a propagation equation for the transverse components $(E_x,E_y,H_x,H_y)$.
The demonstration can be found in reference~\onlinecite{Berreman} and leads to%
\footnote{Note that equation~(3) in Berreman's article is wrong and should read $\mathbf{C=MG}$.}
%
\begin{equation}\label{eq:propagation}
\frac{\partial \Psi}{\partial z} = i k_0 \Delta(z) \Psi(z),
\quad\mathrm{with}\quad
\Psi = 
\begin{pmatrix}
E_x\\
E_y\\
H_x\\
H_y
\end{pmatrix},
\end{equation}
where $\Delta(z)$ is a $4\times4$ matrix.
For a general dielectric tensor, the matrix is \cite{Schubert} $\Delta(z) =$
$$
\renewcommand*{\arraystretch}{2}
\newcommand{\Kx}{\ensuremath{{\scriptstyle K_x}}}
\newcommand{\KxSq}{\ensuremath{{\scriptstyle K_x^2}}}
\begin{pmatrix}
-\Kx \dfrac{\eps_{3,1}}{\eps_{3,3}} & 
-\Kx \dfrac{\eps_{3,2}}{\eps_{3,3}} &
0 & 1 - \dfrac{\KxSq}{\eps_{3,3}}    \\
0 & 0 & -1 & 0 \\
\eps_{2,3} \dfrac{\eps_{3,1}}{\eps_{3,3}} - \eps_{2,1} &
\KxSq - \eps_{2,2} + \eps_{2,3} \dfrac{\eps_{3,2}}{\eps_{3,3}} &
0 & \Kx \dfrac{\eps_{2,3}}{\eps_{3,3}} \\
\eps_{1,1} - \eps_{1,3} \dfrac{\eps_{3,1}}{\eps_{3,3}} &
\eps_{1,2} - \eps_{1,3} \dfrac{\eps_{3,2}}{\eps_{3,3}} & 
0 & -\Kx \dfrac{\eps_{1,3}}{\eps_{3,3}}
\end{pmatrix}
$$
The reduced wave number $K_x = k_x/k_0$ is a constant throughout the sample and depends only on the angle of the incident wave.

For a homogeneous slab $z_1 < z < z_2$, the matrix $\Delta(z)$ is constant and equation~\ref{eq:propagation} can be integrated into $\Psi(z_2) = P_{hs}(z_2,z_1) \times  \Psi(z_1)$, where the propagator is given by the matrix exponential
$$
P_{hs}(z_2,z_1) = 
\exp\big(\displaystyle i\ (z_2-z_1)\ k_0\ \Delta\big)
$$
The numerical computation of a matrix exponential is generally slow.
Berreman suggested to diagonalize the matrix by searching the eigenvalues and eigenvectors \cite{Berreman}.
However, the knowledge of the eigenvectors is not necessary and the result can be expressed based on the eigenvalues only%
\footnote{Knowledge of the eigenvectors is necessary only for the exit transition matrix if the back half-space is anisotropic.}.
%
I.~Abdulhalim \emph{et al.} used the Lagrange-Sylvester interpolation polynomial \cite{1985_Abdulhalim, 1999_Abdulhalim, Gantmakher} and Wöhler \emph{et al.} used Cayley-Hamilton's theorem \cite{1988_Wohler, 1991_Wohler}. 
Both approaches lead to the same expression.
%
The eigenvalues can be calculated numerically as the roots of the characteristic polynomial.
Literal expressions can be found for specific situations, like for a uniaxial material \cite{1988_Wohler, 1991_Wohler} or in the case of normal incidence \cite{1985_Abdulhalim}.
Also, in the case of a diagonal tensor, a specific solution for $P_{hs}$ is available \cite{1999_Abdulhalim}.

If a part of the sample is inhomogeneous, it is subdivided into slices over which the variation of $\Delta(z)$ is small. 
For such a slice, the propagator $P(z_2,z_1)$ is approximated by a homogeneous slab for which the $\Delta$ matrix is evaluated in the middle of the interval,
$$
P(z_2,z_1) \simeq 
\exp\left(i\ (z_2-z_1)\ k_0\ \Delta\left(\frac{z_1+z_2}{2}\right)\right).
$$
The total propagator $P(z_N,z_0)$ for $N$ slices between $z_0$ and $z_N$ is approximated the product
$$P_a(z_N,z_0) = P_{hs}(z_N, z_{N-1})\times \cdots  \times P_{hs}(z_1,z_0).$$
The order of the error is
$$P(z_N,z_0)-P_a(z_N,z_0) = O(1/N^2)$$
and  Z.~Lu demonstrated \cite{2007_Lu} that this does not depend on whether the thin slab propagator $P_{hs}$ is the exact propagator or an approximation, possibly to first order.
Consequently, the simplest solution to equation~\ref{eq:propagation} is to take the first order expansion
$$
\Psi(z_2) \simeq 
i (z_2-z_1) k_0\ \Delta\left(\frac{z_1+z_2}{2}\right)\times \Psi(z_1),
$$
which corresponds to the first order expansion of the exponential,
$$
P_{hs}(z_2,z_1) \simeq i (z_2-z_1) k_0\ \Delta\left(\frac{z_1+z_2}{2}\right)
$$
For improving convergence and efficiency, Z.~Lu presented an extrapolation method to eliminate the leading terms of the error \cite{2007_Lu}.
If the propagator used for the thin homogeneous slabs is the exact propagator $P_{hs}$, the error is reduced to $O(1/N^4)$.
Z.~Lu presented another version with a symplectic integrator that showed improved convergence \cite{2010_Lu}.
In this version, the propagator for a thin slab is simply the product of three homogeneous slab propagators evaluated for different thickness and position (see equation~(10) in the reference).

\section{Transition to half-spaces}

For isotropic half-spaces, the relation between the waves and the vector $\Psi$ at the boundary is given by the transition matrices $L_f$ and $L_b$,
$$
\Psi(0) = L_f 
\begin{pmatrix}
E_{is}\\
E_{rs}\\
E_{ip}\\
E_{rp}
\end{pmatrix}_{z=z_f}
\quad\mathrm{and} \quad 
\Psi(z_b) = L_b
\begin{pmatrix}
E_{ts}\\
0 \\
E_{tp}\\
0
\end{pmatrix}_{z=z_b}
$$
The ordering of the components is $(s^+,s^-,p^+,p^-)$, where the superscript indicates the traveling direction of the wave along the $z$ axis.
If the angle of the wave traveling in the $z$ direction is $\phi$ and the refractive index of the half-space is $n$, we have
$$
L = 
\begin{pmatrix}
0 & 0 & \cos(\phi) & \cos(\phi) \\
1 & 1 & 0 & 0 \\
-n \cos(\phi) & n \cos(\phi) & 0 & 0 \\
0 & 0 & n & -n 
\end{pmatrix}
$$

When the back half-space is anisotropic, we can not decompose the transmitted wave on $s$ and $p$ polarizations, but by analogy, we decompose $\Psi(z_b)$ over the eigenvectors $\Psi_k$ of the $\Delta_b$ matrix, 
$$
\Psi(z_b)= \sum_{k=1}^4 c_k \Psi_k
$$
We sort the eigenvectors so that $\Psi_1$ and $\Psi_3$ correspond to waves propagating in the $z$ direction, i.e. the eigenvalues $q_1$ and $q_3$ have positive real part.
This description incorporates the isotropic case for which $(c_1,c_3) = (E_{ts}, E_{tp})(z_b)$.
We can write 
$$
\Psi(z_b) = L_b
\begin{pmatrix}
c_1\\
0 \\
c_3\\
0
\end{pmatrix}
\quad\mathrm{with}\quad
L_b =
\begin{pmatrix} 
\Psi_{1_1} & 0 & \Psi_{3_1} & 0 \\ 
\Psi_{1_2} & 0 & \Psi_{3_2} & 0 \\ 
\Psi_{1_3} & 0 & \Psi_{3_3} & 0 \\ 
\Psi_{1_4} & 0 & \Psi_{3_4} & 0 \\ 
\end{pmatrix}
$$

\section{Matrix assembling and Jones matrices}

The global propagation matrix and the two transition matrices are assembled in order to relate the coefficients of the waves in the two half-spaces.
We obtain the total transfer matrix $T$ with
$$
\begin{pmatrix}
E_{is}\\
E_{rs}\\
E_{ip}\\
E_{rp}
\end{pmatrix}_{z_f}
\!\!
= 
\ \ 
L_f^{-1}\ P(z_f, z_b)\ L_b \ 
\begin{pmatrix}
c_1\\ 0 \\ c_3\\ 0
\end{pmatrix} 
\ \ 
= 
\ \ 
T \ 
\begin{pmatrix}
c_1\\ 0 \\ c_3\\ 0
\end{pmatrix}.
$$
The two useful relations can be extracted,
\begin{align*}
& 
\begin{pmatrix}
E_{ip} \\ E_{is}
\end{pmatrix}_{z_f} \!\! = \ \ 
\begin{pmatrix}
T_{33} & T_{31} \\
T_{13} & T_{11} 
\end{pmatrix}
\begin{pmatrix}
c_3 \\ c_1
\end{pmatrix} = \ T_{it} \ 
\begin{pmatrix}
c_3 \\ c_1
\end{pmatrix}\hphantom{.} 
\\[3pt]
&
\begin{pmatrix}
E_{rp} \\ E_{rs} 
\end{pmatrix}_{z_f} \!\! = \ \ 
\begin{pmatrix}
T_{43} & T_{41} \\
T_{23} & T_{21} 
\end{pmatrix}
\begin{pmatrix}
c_3\\ c_1
\end{pmatrix} =\ T_{rt} \ 
\begin{pmatrix}
c_3\\ c_1
\end{pmatrix}.
\end{align*}

Reflection of the incident wave can be described by a Jones matrix $T_{ri}$, and if the back half-space is isotropic, a Jones matrix $T_{ti}$ for transmission can also be defined \cite{Jones, Fujiwara},
\begin{align*}
\begin{pmatrix}
E_{rp} \\ E_{rs}
\end{pmatrix}_{z_f} \!\! = \ \ 
\begin{pmatrix}
r_{pp} & r_{ps} \\
r_{sp} & r_{ss} 
\end{pmatrix}
\begin{pmatrix}
E_{ip} \\ E_{is}
\end{pmatrix}_{z_f} \!\! = \ \ T_{ri} \ 
\begin{pmatrix}
E_{ip} \\ E_{is}
\end{pmatrix}_{z_f}
\\[3pt]
\begin{pmatrix}
E_{tp} \\ E_{ts}
\end{pmatrix}_{z_b} \!\! = \ \ 
\begin{pmatrix}
t_{pp} & t_{ps} \\
t_{sp} & t_{ss} 
\end{pmatrix}
\begin{pmatrix}
E_{ip} \\ E_{is}
\end{pmatrix}_{z_f} \!\! = \ \ T_{ti} \ 
\begin{pmatrix}
E_{ip} \\ E_{is}
\end{pmatrix}_{z_f}
\end{align*}
These matrices contain all the information on reflection and transmission of the sample. 
They are obtained by the relations 
$$
T_{ri} = T_{rt} T_{it}^{-1} 
\quad\mathrm{and}\quad 
T_{ti} = T_{it}^{-1}.
$$

\section{Circularly polarized light}
When fields are decomposed over the $s$ and $p$ polarizations, the basis for the Jones vectors is $(\mathbf{E}_{p}, \mathbf{E}_{s})$ with
$$
\mathbf{E}_{p} = \begin{pmatrix} 1 \\ 0 \end{pmatrix}
\quad\mathrm{and}\quad
\mathbf{E}_{s} = \begin{pmatrix} 0 \\ 1 \end{pmatrix}.
$$
It is possible to consider other bases, for example the left and right circular polarizations.
For the incident and transmitted waves the Jones vectors are
$$
\mathbf{E}_{iL}, \mathbf{E}_{tL}  = \frac{1}{\sqrt{2}} \begin{pmatrix} 1 \\ i \end{pmatrix}
\quad\mathrm{and}\quad
\mathbf{E}_{iR},\mathbf{E}_{tR}  = \frac{1}{\sqrt{2}} \begin{pmatrix} 1 \\ -i \end{pmatrix}.
$$
For the reflected wave, we have
$$
\mathbf{E}_{rL} = \frac{1}{\sqrt{2}} \begin{pmatrix} 1 \\ -i \end{pmatrix}
\quad\mathrm{and}\quad
\mathbf{E}_{rR} = \frac{1}{\sqrt{2}} \begin{pmatrix} 1 \\ i \end{pmatrix}.
$$
The transformation matrix from the $(s,p)$ basis to the $(L,R)$ basis will be called $C$ in the case of incident and reflected waves, and $D$ for the reflected wave. 
We have
$$
C = \frac{1}{\sqrt{2}} 
\begin{pmatrix} 
1 &  1 \\ 
i & -i \\ 
\end{pmatrix}
\quad\mathrm{and}\quad
D = \frac{1}{\sqrt{2}} 
\begin{pmatrix} 
1 &  1 \\ 
-i & i \\ 
\end{pmatrix}.
$$
The relations between the Jones vectors in the two bases are 
\begin{eqnarray*}
\begin{pmatrix} E_{ip} \\ E_{is} \end{pmatrix}
= C\  
\begin{pmatrix} E_{iL} \\ E_{iR} \end{pmatrix}
& , &
\begin{pmatrix} E_{tp} \\ E_{is} \end{pmatrix}
= C\  
\begin{pmatrix} E_{tL} \\ E_{iR} \end{pmatrix},\\
\mathrm{and} & &
\begin{pmatrix} E_{rp} \\ E_{is} \end{pmatrix}
= D\  
\begin{pmatrix} E_{rL} \\ E_{iR} \end{pmatrix}.
\end{eqnarray*}
As a result, the Jones matrices for circularly polarized light are
$T^c_{ti} = C^{-1}\ T_{ti}\ C$ and
$T^c_{ri} = D^{-1}\ T_{ri}\ C$.

\section{Ellipsometry parameters}

Ellipsometry parameters describe the reflection of the sample by the normalized reflection matrix $T_{ri}/r_{ss}$, as presented in Fujiwara's book, reference \onlinecite{Fujiwara}, p.~220.
However, since we use an opposite orientation convention for $E_{rp}$, a change of sign is necessary for matching the convention of ellipsometry and we define
\begin{multline*}
\begin{pmatrix}
\tan(\psi_{pp})\ e^{\displaystyle -i\Delta_{pp}} & 
\tan(\psi_{ps})\ e^{\displaystyle -i\Delta_{ps}} \\
\tan(\psi_{sp})\ e^{\displaystyle -i\Delta_{sp}} &
1
\end{pmatrix}\\
=
\begin{pmatrix}
-r_{pp}/r_{ss} & -r_{ps}/r_{ss} \\
\hphantom{-}r_{sp}/r_{ss} & 1 
\end{pmatrix}.
\end{multline*}
The minus sign in front of $\Delta$ is chosen due to the $\exp(-i\omega t)$ phase convention.
The ellipsometry angles are chosen with $\psi\in[0,\pi/2]$ and $\Delta\in\ ]-\pi,\pi]$.
If the sample is isotropic in the $(x,y)$ direction, the off-diagonal coefficients $sp$ and $ps$ vanish and only two parameters are needed to describe the reflection.
We have
$$
T_{ri} =
\begin{pmatrix}
r_{pp} & 0\\
0 & r_{ss}
\end{pmatrix}
\quad\mathrm{and}\quad
T_{ti} =
\begin{pmatrix}
t_{pp} & 0\\
0 & t_{ss}
\end{pmatrix}
$$
and we define the ellipsometry parameters $\psi$ and $\Delta$ with 
$$
\tan(\psi)\ e^{\displaystyle -i\Delta} = -r_{pp}/r_{ss}.
$$

\section{Installation and use}

Berreman4x4 is offered as a Python module named \verb/Berreman4x4.py/, which can be imported and used in Python scripts with the command 
\begin{verbatim}
import Berreman4x4
\end{verbatim}
The module file should either be present in the working directory or be accessible in the module path.
A convenient organization is to store Python modules in a special directory pointed by the \verb/PYTHONPATH/ environment variable.
For example, my \verb/.bashrc/ script contains
\begin{verbatim}
export PYTHONPATH="/home/castany/.python"
\end{verbatim}
and the \verb/.python/ directory contains symbolic links to the different Python modules I use.

Berreman4x4 depends on the standard Python modules NumPy\cite{NumPy} and SciPy\cite{SciPy}.
Application examples need module matplotlib\cite{matplotlib} for plotting.

General workflow with Berreman4x4 consists of  
(1) building the structure, 
(2) calculating Jones matrices, 
(3) extracting the desired coefficients and plotting.

\section{Code documentation and examples}

\begin{figure*}
\includegraphics[width=\textwidth, page=1]{fig/UML-crop}
\caption{\label{fig:class1}Class diagram of Berreman4x4: structure, materials, and module functions. }
\end{figure*}

\begin{figure*}
\includegraphics[width=\textwidth, page=2]{fig/UML-crop}
\caption{\label{fig:class2}Class diagram of Berreman4x4: layers, inhomogeneous materials, and half-spaces.}
\end{figure*}

The source code contains detailed information on the classes and functions, incorporated as docstrings.
They can be conveniently displayed when working with a shell like IPython \cite{IPython}.

Commented examples are bundled with the program and can be run directly from the command line.
They range from simple situations to more complex structures: reflection on an interface, interferences in a thin layer, reflection on a Bragg mirror, 90$^\circ$ twisted nematic liquid crystal.
Whenever possible, result from Berreman4x4 is compared with exact analytical result, and show excellent match.
The code is also validated against results presented in reference~\onlinecite{Fujiwara}.

The UML class diagram of Berreman4x4 is presented on figures~\ref{fig:class1} and \ref{fig:class2}.
A \texttt{Structure} is built from a list of \texttt{Layer} objects, between one front and one back \texttt{HalfSpace} objects.
\texttt{Material} objects are created for defining \texttt{MaterialLayer} objects.
Several classes of non-dispersive materials are provided, but dispersive materials may also be created, like the class \texttt{IsotropicDispersive}.
Additional classes of materials can be created by deriving classes \texttt{Material} and \texttt{DispersionLaw}.
Layers with a continuous spatial variation of the permittivity tensor are described with an \texttt{InhomogeneousMaterial} and form an \texttt{InhomogeneousLayer} object.

\section{TODO}

General:
\begin{itemize}
\item Check Berreman's equations again. Which are the assumptions?
\item Verify efficiency of Lu's method. It seems to be worse than the midpoint method for the twisted nematic example. Strange.
\end{itemize}
%
Source code:
\begin{itemize}
\item Homogeneous layer: implement exact solution with Cayley-Hamilton or Lagrange-Sylvester.
\item Calculation of the fields inside the structure.
\end{itemize}


