% Encoding: utf-8

\documentclass[a4paper, 10pt, oneside, twocolumn, openany]{memoir}

\usepackage{lipsum} % for tests

\usepackage{lmodern}
\usepackage[T1]{fontenc}
\usepackage[utf8]{inputenc}

\usepackage{textcomp}
\usepackage{amsmath}
\usepackage{amssymb}

\usepackage[pdftex]{graphicx}
\usepackage[colorlinks, pdftex]{hyperref}
\hypersetup{pdftitle={Berreman4x4}}

\usepackage[numbers, super, sort&compress]{natbib}

\settypeblocksize{56pc}{42.5pc}{*}  % text body
\setlrmargins{*}{*}{0.7}            % right / left margin ratio
\setulmargins{*}{*}{2}              % upper / lower margin ratio
\setcolsepandrule{1.5pc}{0pt}       % two column separation and rule
\checkandfixthelayout

\chapterstyle{komalike}
\setlength{\beforechapskip}{0pt}    % vertical space added above chapter name
\setcounter{secnumdepth}{3}

% Numbering without chapter number
\renewcommand*{\thesection}{\arabic{section}}
\renewcommand*{\theequation}{\arabic{equation}}
\renewcommand*{\thefigure}{\arabic{figure}}

% Special command for chapters
\newcommand{\chapterdoc}[1]{\chapter*{#1}%
    \addtocounter{chapter}{1}%
    \setcounter{section}{0}
}

% Chapterprecis settings
\makeatletter
\renewcommand{\prechapterprecis}{%
    \vspace*{\prechapterprecisshift}%
    \begingroup\precisfont}
\renewcommand{\postchapterprecis}{\endgroup\vspace{1em}}
\newcommand{\chapterauthor}[1]{\chapterprecis{#1}
    \@afterindentfalse\@afterheading}
\makeatother
\renewcommand*{\precisfont}{\normalfont\sffamily}


% Provide command name \onlinecite{}
\def\onlinecite{\citenum}

% Caption for figures
\renewcommand*{\figurename}{FIG.}

% Useful definitions
\newcommand{\eps}{\ensuremath{\varepsilon}}

% A few changes in the default commands
\renewcommand{\Re}{\mathop{\mathrm{Re}}}
\renewcommand{\Im}{\mathop{\mathrm{Im}}}





\begin{document}

\begin{titlingpage}

\pretitle{\begin{center}\Huge\sffamily\bfseries}
\posttitle{\par\end{center}\vskip 3.5cm}

\preauthor{\large \lineskip 0.5em \begin{tabular}[t]{l}}
\postauthor{\end{tabular}\par}

\predate{\large \lineskip 0.5em \begin{tabular}[t]{l}}
\postdate{\end{tabular}\par\vskip 1cm}

\usethanksrule
\thanksheadextra{(}{)}
\thanksmarkstyle{\textsuperscript{(#1)}}
\thanksmarkseries{alph}

\setlength{\absparindent}{0pt}
% Use same alignment for abstrat title than for abstrat text
\makeatletter\renewcommand{\absnamepos}{@bstr@ctlist}\makeatother
\renewcommand{\abstractnamefont}{\normalfont\bfseries}
\renewcommand{\abstracttextfont}{\normalfont}

\title{Berreman4x4}

\author{\textsf{Olivier Castany}%
    \thanks{Electronic mail: 
    \href{mailto:olivier.castany@telecom-bretagne.eu}%
         {olivier.castany@telecom-bretagne.eu}}\\
    \emph{Department of Optics, Telecom Bretagne, 29238 Brest, France}
}

\date{\today}

\maketitle

\begin{abstract}
Electromagnetic plane wave propagation in stratified anisotropic media was described by Berreman with the use of 4$\times$4 matrices.
Berreman4x4 is a numerical implementation of the method in Python.
Examples of applications are ellipsometry analysis, design of Bragg mirrors or study of twisted liquid crystal structures.
\end{abstract}

\end{titlingpage}


%%%%%%%%%%%%%%%%%%%%%%%%%%%%%%%%%%%%%%%%%%%%%%%%%%%%%%%%%%%%%%%%%%%%%%%%%%%%%

% Encoding: utf-8

\chapterdoc{Description of Berreman4x4}
\chapterauthor{Olivier Castany}

Electromagnetic wave propagation in stratified media is important in several applications like ellipsometry analysis, Bragg mirrors or twisted liquid crystal structures.
Propagation of plane waves in isotropic media can be solved with 2$\times$2 propagation matrix methods \cite{BornWolf}.
In the case of anisotropic layers, a 4$\times$4 propagation matrix method was developed \cite{1970_TeilerHenvis, Berreman} and is now known as Berreman's method. 

The present work is an open-source implementation of the method in Python.
This programming language was chosen for ease of use, readability of the source code, portability and availability of scientific libraries (NumPy and SciPy).
A drawback is slow speed, because the language is interpreted%
\footnote{%
High speed calculations are possible with NumPy and SciPy if array operations are vectorized.
However the structure of Berreman4x4 is difficult to vectorize with enough generality.}.

Calculations are based on articles from D.~Berreman \cite{Berreman} (1972) and M.~Schubert \cite{Schubert} (1996).
Application to ellipsometry is based on the book from H.~Fujiwara \cite{Fujiwara} (2007).
General references for optics in anisotropic media are found in the books from M.~Born and E.~Wolf \cite{BornWolf} and J.~Jackson \cite{Jackson}.

\section{Presentation}

As described on figure~\ref{fig:geometry}, we consider a stratified sample with layers invariant in the $(x,y)$ plane and stacked in the $z$ direction, starting from $z_f=0$.
Because of the translation invariance, modes with bounded fields can be classified according to wave numbers $(k_x,k_y)\in\mathbb{R}^2$.
Without loss of generality, we consider plane waves in the $(x,z)$ direction, i.e. $k_y=0$.

The front half-space is isotropic and a plane wave can be decomposed into $s$ and $p$ polarizations.
The $s$ polarization is a wave with electric field perpendicular (\emph{senkrecht}) to the plane of incidence, i.e. along $y$.
The $p$ polarization is a wave with electric field parallel to the plane of incidence.
A plane wave $i$ is incident from the front half-space with incidence angle $\phi_i$ and reflected into a plane wave $r$ with the same angle.
Angles are oriented by the $y$ direction.
In the general case, the back half-space is anisotropic and two transmitted plane waves $t$ are induced, with the same $k_x$, but different angles $\phi_t$.
The electric vector is not necessarily perpendicular to $\vec k$ and $s$ and $p$ waves can not be considered.

\begin{figure}
\includegraphics[width=\columnwidth]{fig/geometry}
\caption{\label{fig:geometry}Geometry of the sample, with input and output plane waves. }
\end{figure}

\section{Conventions and units}

Formulas are mostly based on M.~Schubert's article \cite{Schubert}, with one difference being the orientation of the incident $p$ polarization.
M.~Schubert takes the convention used in ellipsometry, where the base electric vectors point towards the sample in the front and exit half spaces.
In our work, we used a different convention, in which the base electric vectors point towards the $x$ direction in both half-spaces.
With this convention, the base electric vectors for the incident, reflected and transmitted waves are all in the same direction in the case of reflection with normal incidence.

Gaussian units are used in the detail of the calculation.
This makes it easier to compare our results with past literature, which uses Gaussian units for the most part.
Of course, reflection and transmission coefficients have no unit and do not depend on the choice of unit.
Conversion between units is presented in the appendix on electromagnetic units in reference~\onlinecite{Jackson}.
In Gaussian units, Maxwell's equations read
$$
\nabla\times E = -\frac{1}{c} \frac{\partial B}{\partial t} 
\quad \mathrm{and} \quad
\nabla\times H = \frac{1}{c} \frac{\partial D}{\partial t}.
$$
We consider linear materials with the constitutive relations
$D = \varepsilon E$ and $B = \mu H$.
In Gaussian units, $\varepsilon_0=\mu_0=1$ and the relation to S.I. units is
$$
E_\mathrm{S.I.} = \frac{E}{\sqrt{4\pi\varepsilon_0}}
\quad \mathrm{and} \quad
H_\mathrm{S.I.} = \frac{H}{\sqrt{4\pi\mu_0}},
$$
where $\varepsilon_0$ and $\mu_0$ are the usual S.I. values.
We will consider non-magnetic materials, i.e. $\mu = 1$.
%
The convention for time-varying complex fields is taken as $\exp(-i\omega t)$, and Maxwell's equations read
$$
\nabla\times E = k_0 H
\quad \mathrm{and} \quad
\nabla\times H = -k_0 \varepsilon E,
$$
with $k_0=\omega/c$.

\section{Propagation inside the sample}

In the stratified sample, Maxwell's equations lead to a propagation equation for the transverse components $(E_x,E_y,H_x,H_y)$.
The demonstration can be found in reference~\onlinecite{Berreman} and leads to%
\footnote{Note that equation~(3) in Berreman's article is wrong and should read $\mathbf{C=MG}$.}
%
\begin{equation}\label{eq:propagation}
\frac{\partial \Psi}{\partial z} = i k_0 \Delta(z) \Psi(z),
\quad\mathrm{with}\quad
\Psi = 
\begin{pmatrix}
E_x\\
E_y\\
H_x\\
H_y
\end{pmatrix},
\end{equation}
where $\Delta(z)$ is a $4\times4$ matrix.
For a general dielectric tensor, the matrix is \cite{Schubert} $\Delta(z) =$
$$
\renewcommand*{\arraystretch}{2}
\newcommand{\Kx}{\ensuremath{{\scriptstyle K_x}}}
\newcommand{\KxSq}{\ensuremath{{\scriptstyle K_x^2}}}
\begin{pmatrix}
-\Kx \dfrac{\eps_{3,1}}{\eps_{3,3}} & 
-\Kx \dfrac{\eps_{3,2}}{\eps_{3,3}} &
0 & 1 - \dfrac{\KxSq}{\eps_{3,3}}    \\
0 & 0 & -1 & 0 \\
\eps_{2,3} \dfrac{\eps_{3,1}}{\eps_{3,3}} - \eps_{2,1} &
\KxSq - \eps_{2,2} + \eps_{2,3} \dfrac{\eps_{3,2}}{\eps_{3,3}} &
0 & \Kx \dfrac{\eps_{2,3}}{\eps_{3,3}} \\
\eps_{1,1} - \eps_{1,3} \dfrac{\eps_{3,1}}{\eps_{3,3}} &
\eps_{1,2} - \eps_{1,3} \dfrac{\eps_{3,2}}{\eps_{3,3}} & 
0 & -\Kx \dfrac{\eps_{1,3}}{\eps_{3,3}}
\end{pmatrix}
$$
The reduced wave number $K_x = k_x/k_0$ is a constant throughout the sample and depends only on the angle of the incident wave.

For a homogeneous slab $z_1 < z < z_2$, the matrix $\Delta(z)$ is constant and equation~\ref{eq:propagation} can be integrated into $\Psi(z_2) = P_{hs}(z_2,z_1) \times  \Psi(z_1)$, where the propagator is given by the matrix exponential
$$
P_{hs}(z_2,z_1) = 
\exp\big(\displaystyle i\ (z_2-z_1)\ k_0\ \Delta\big)
$$
The numerical computation of a matrix exponential is generally slow.
Berreman suggested to diagonalize the matrix by searching the eigenvalues and eigenvectors \cite{Berreman}.
However, the knowledge of the eigenvectors is not necessary and the result can be expressed based on the eigenvalues only%
\footnote{Knowledge of the eigenvectors is necessary only for the exit transition matrix if the back half-space is anisotropic.}.
%
I.~Abdulhalim \emph{et al.} used the Lagrange-Sylvester interpolation polynomial \cite{1985_Abdulhalim, 1999_Abdulhalim, Gantmakher} and Wöhler \emph{et al.} used Cayley-Hamilton's theorem \cite{1988_Wohler, 1991_Wohler}. 
Both approaches lead to the same expression.
%
The eigenvalues can be calculated numerically as the roots of the characteristic polynomial.
Literal expressions can be found for specific situations, like for a uniaxial material \cite{1988_Wohler, 1991_Wohler} or in the case of normal incidence \cite{1985_Abdulhalim}.
Also, in the case of a diagonal tensor, a specific solution for $P_{hs}$ is available \cite{1999_Abdulhalim}.

If a part of the sample is inhomogeneous, it is subdivided into slices over which the variation of $\Delta(z)$ is small. 
For such a slice, the propagator $P(z_2,z_1)$ is approximated by a homogeneous slab for which the $\Delta$ matrix is evaluated in the middle of the interval,
$$
P(z_2,z_1) \simeq 
\exp\left(i\ (z_2-z_1)\ k_0\ \Delta\left(\frac{z_1+z_2}{2}\right)\right).
$$
The total propagator $P(z_N,z_0)$ for $N$ slices between $z_0$ and $z_N$ is approximated the product
$$P_a(z_N,z_0) = P_{hs}(z_N, z_{N-1})\times \cdots  \times P_{hs}(z_1,z_0).$$
The order of the error is
$$P(z_N,z_0)-P_a(z_N,z_0) = O(1/N^2)$$
and  Z.~Lu demonstrated \cite{2007_Lu} that this does not depend whether the thin slab propagator $P_{hs}$ is the exact propagator or an approximation, possibly to first order.
Consequently, the simplest solution to equation~\ref{eq:propagation} is to take the first order expansion
$$
\Psi(z_2) \simeq 
i (z_2-z_1) k_0\ \Delta\left(\frac{z_1+z_2}{2}\right)\times \Psi(z_1),
$$
which corresponds to the first order expansion of the exponential,
$$
P_{hs}(z_2,z_1) \simeq i (z_2-z_1) k_0\ \Delta\left(\frac{z_1+z_2}{2}\right)
$$
For improving convergence and efficiency, Z.~Lu presented an extrapolation method to eliminate the leading terms of the error \cite{2007_Lu}.
If the propagator used for the thin homogeneous slabs is the exact propagator $P_{hs}$, the error is reduced to $O(1/N^4)$.
Z.~Lu presented another version with a symplectic integrator that showed improved convergence \cite{2010_Lu}.
In this version, the propagator for a thin slab is simply the product of three homogeneous slab propagators evaluated for different thickness and position (see equation~(10) in the reference).

\section{Transition to half-spaces}

For isotropic half-spaces, the relation between the waves and the vector $\Psi$ at the boundary is given by the transition matrices $L_f$ and $L_b$,
$$
\Psi(0) = L_f 
\begin{pmatrix}
E_{is}\\
E_{rs}\\
E_{ip}\\
E_{rp}
\end{pmatrix}_{z=z_f}
\quad\mathrm{and} \quad 
\Psi(z_b) = L_b
\begin{pmatrix}
E_{ts}\\
0 \\
E_{tp}\\
0
\end{pmatrix}_{z=z_b}
$$
The ordering of the components is $(s^+,s^-,p^+,p^-)$, where the superscript indicates the traveling direction of the wave along the $z$ axis.
If the angle of the wave traveling in the $z$ direction is $\phi$ and the refractive index of the half-space is $n$, we have
$$
L = 
\begin{pmatrix}
0 & 0 & \cos(\phi) & \cos(\phi) \\
1 & 1 & 0 & 0 \\
-n \cos(\phi) & n \cos(\phi) & 0 & 0 \\
0 & 0 & n & -n 
\end{pmatrix}
$$

When the back half-space is anisotropic, we can not decompose the transmitted wave on $s$ and $p$ polarizations, but by analogy, we decompose $\Psi(z_b)$ over the eigenvectors $\Psi_k$ of the $\Delta_b$ matrix, 
$$
\Psi(z_b)= \sum_{k=1}^4 c_k \Psi_k
$$
We sort the eigenvectors so that $\Psi_1$ and $\Psi_3$ correspond to waves propagating in the $z$ direction, i.e. the eigenvalues $q_1$ and $q_3$ have positive real part.
This description incorporates the isotropic case for which $(c_1,c_3) = (E_{ts}, E_{tp})(z_b)$.
We can write 
$$
\Psi(z_b) = L_b
\begin{pmatrix}
c_1\\
0 \\
c_3\\
0
\end{pmatrix}
\quad\mathrm{with}\quad
L_b =
\begin{pmatrix} 
\Psi_{1_1} & 0 & \Psi_{3_1} & 0 \\ 
\Psi_{1_2} & 0 & \Psi_{3_2} & 0 \\ 
\Psi_{1_3} & 0 & \Psi_{3_3} & 0 \\ 
\Psi_{1_4} & 0 & \Psi_{3_4} & 0 \\ 
\end{pmatrix}
$$

\section{Matrix assembling and Jones matrices}

The global propagation matrix and the two transition matrices are assembled in order to relate the coefficients of the waves in the two half-spaces.
We obtain the total transfer matrix $T$ with
$$
\begin{pmatrix}
E_{is}\\
E_{rs}\\
E_{ip}\\
E_{rp}
\end{pmatrix}_{z_f}
\!\!
= 
\ \ 
L_f^{-1}\ P(z_f, z_b)\ L_b \ 
\begin{pmatrix}
c_1\\ 0 \\ c_3\\ 0
\end{pmatrix} 
\ \ 
= 
\ \ 
T \ 
\begin{pmatrix}
c_1\\ 0 \\ c_3\\ 0
\end{pmatrix}.
$$
The two useful relations can be extracted,
\begin{align*}
& 
\begin{pmatrix}
E_{ip} \\ E_{is}
\end{pmatrix}_{z_f} \!\! = \ \ 
\begin{pmatrix}
T_{33} & T_{31} \\
T_{13} & T_{11} 
\end{pmatrix}
\begin{pmatrix}
c_3 \\ c_1
\end{pmatrix} = \ T_{it} \ 
\begin{pmatrix}
c_3 \\ c_1
\end{pmatrix}\hphantom{.} 
\\[3pt]
&
\begin{pmatrix}
E_{rp} \\ E_{rs} 
\end{pmatrix}_{z_f} \!\! = \ \ 
\begin{pmatrix}
T_{43} & T_{41} \\
T_{23} & T_{21} 
\end{pmatrix}
\begin{pmatrix}
c_3\\ c_1
\end{pmatrix} =\ T_{rt} \ 
\begin{pmatrix}
c_3\\ c_1
\end{pmatrix}.
\end{align*}

Reflection of the incident wave can be described by a Jones matrix $T_{ri}$, and if the back half-space is isotropic, a Jones matrix $T_{ti}$ for transmission can also be defined \cite{Jones, Fujiwara},
\begin{align*}
\begin{pmatrix}
E_{rp} \\ E_{rs}
\end{pmatrix}_{z_f} \!\! = \ \ 
\begin{pmatrix}
r_{pp} & r_{ps} \\
r_{sp} & r_{ss} 
\end{pmatrix}
\begin{pmatrix}
E_{ip} \\ E_{is}
\end{pmatrix}_{z_f} \!\! = \ \ T_{ri} \ 
\begin{pmatrix}
E_{ip} \\ E_{is}
\end{pmatrix}_{z_f}
\\[3pt]
\begin{pmatrix}
E_{tp} \\ E_{ts}
\end{pmatrix}_{z_b} \!\! = \ \ 
\begin{pmatrix}
t_{pp} & t_{ps} \\
t_{sp} & t_{ss} 
\end{pmatrix}
\begin{pmatrix}
E_{ip} \\ E_{is}
\end{pmatrix}_{z_f} \!\! = \ \ T_{ti} \ 
\begin{pmatrix}
E_{ip} \\ E_{is}
\end{pmatrix}_{z_f}
\end{align*}
These matrices contain all the information on reflexion and transmission of the sample. 
They are obtained by the relations 
$$
T_{ri} = T_{rt} T_{it}^{-1} 
\quad\mathrm{and}\quad 
T_{ti} = T_{it}^{-1}.
$$

\section{Circularly polarized light}
When fields are decomposed over the $s$ and $p$ polarizations, the basis for the Jones vectors is $(\mathbf{E}_{p}, \mathbf{E}_{s})$ with
$$
\mathbf{E}_{p} = \begin{pmatrix} 1 \\ 0 \end{pmatrix}
\quad\mathrm{and}\quad
\mathbf{E}_{s} = \begin{pmatrix} 0 \\ 1 \end{pmatrix}.
$$
It is possible to consider other bases, for example the left and right circular polarizations.
For the incident and transmitted waves the Jones vectors are
$$
\mathbf{E}_{iL}, \mathbf{E}_{tL}  = \frac{1}{\sqrt{2}} \begin{pmatrix} 1 \\ i \end{pmatrix}
\quad\mathrm{and}\quad
\mathbf{E}_{iR},\mathbf{E}_{tR}  = \frac{1}{\sqrt{2}} \begin{pmatrix} 1 \\ -i \end{pmatrix}.
$$
For the reflected wave, we have
$$
\mathbf{E}_{rL} = \frac{1}{\sqrt{2}} \begin{pmatrix} 1 \\ -i \end{pmatrix}
\quad\mathrm{and}\quad
\mathbf{E}_{rR} = \frac{1}{\sqrt{2}} \begin{pmatrix} 1 \\ i \end{pmatrix}.
$$
The transformation matrix from the $(s,p)$ basis to the $(L,R)$ basis will be called $C$ in the case of incident and reflected waves, and $D$ for the reflected wave. 
We have
$$
C = \frac{1}{\sqrt{2}} 
\begin{pmatrix} 
1 &  1 \\ 
i & -i \\ 
\end{pmatrix}
\quad\mathrm{and}\quad
D = \frac{1}{\sqrt{2}} 
\begin{pmatrix} 
1 &  1 \\ 
-i & i \\ 
\end{pmatrix}.
$$
The relations between the Jones vectors in the two bases are 
\begin{eqnarray*}
\begin{pmatrix} E_{ip} \\ E_{is} \end{pmatrix}
= C\  
\begin{pmatrix} E_{iL} \\ E_{iR} \end{pmatrix}
& , &
\begin{pmatrix} E_{tp} \\ E_{is} \end{pmatrix}
= C\  
\begin{pmatrix} E_{tL} \\ E_{iR} \end{pmatrix},\\
\mathrm{and} & &
\begin{pmatrix} E_{rp} \\ E_{is} \end{pmatrix}
= D\  
\begin{pmatrix} E_{rL} \\ E_{iR} \end{pmatrix}.
\end{eqnarray*}
As a result, the Jones matrices for circularly polarized light are
$T^c_{ti} = C^{-1}\ T_{ti}\ C$ and
$T^c_{ri} = D^{-1}\ T_{ri}\ C$.

\section{Ellipsometry parameters}

Ellipsometry parameters describe the reflection of the sample by the normalized reflection matrix $T_{ri}/r_{ss}$ (see reference~\onlinecite{Fujiwara}, p.~220).
However, since we use an opposite orientation convention for $E_{rp}$, a change of sign is necessary for matching the convention of ellipsometry and we have the definition
\begin{multline*}
\begin{pmatrix}
-r_{pp}/r_{ss} & -r_{ps}/r_{ss} \\
\hphantom{-}r_{sp}/r_{ss} & 1 
\end{pmatrix} \\
=
\begin{pmatrix}
\tan(\psi_{pp})\ e^{\displaystyle -i\Delta_{pp}} & 
\tan(\psi_{ps})\ e^{\displaystyle -i\Delta_{ps}} \\
\tan(\psi_{sp})\ e^{\displaystyle -i\Delta_{sp}} &
1
\end{pmatrix}.
\end{multline*}
The minus sign in front of $\Delta$ is chosen due to the $\exp(-i\omega t)$ phase convention.
The ellipsometry angles are chosen with $\psi\in[0,\pi/2]$ and $\Delta\in\ ]-\pi,\pi]$.
If the sample is isotropic in the $(x,y)$ direction, the off-diagonal coefficients $sp$ and $ps$ vanish and only two parameters are needed to describe the reflection.
We have
$$
T_{ri} =
\begin{pmatrix}
r_{pp} & 0\\
0 & r_{ss}
\end{pmatrix}
\quad\mathrm{and}\quad
T_{ti} =
\begin{pmatrix}
t_{pp} & 0\\
0 & t_{ss}
\end{pmatrix}
$$
and we define the ellipsometry parameters $\psi$ and $\Delta$ with 
$$
-r_{pp}/r_{ss} = \tan(\psi)\ e^{\displaystyle -i\Delta}.
$$

\section{Installation and use}

Berreman4x4 is offered as a Python module named \verb/Berreman4x4.py/, which can be imported and used in any Python script with the command 
\begin{verbatim}
import Berreman4x4
\end{verbatim}
The module file should either be present in the working directory, or be accessible in the module path.
A convenient organization is to store Python modules in a special directory pointed by the \verb/PYTHONPATH/ environment variable.
For example, my \verb/.bashrc/ script contains
\begin{verbatim}
export PYTHONPATH="/home/castany/.python"
\end{verbatim}
and the \verb/.python/ directory contains symbolic links to the different Python modules I use. 

Berreman4x4 depends on the standard Python modules NumPy and SciPy.
Application examples also use module matplotlib for plotting \cite{NumPy, SciPy, matplotlib}.

\section{Code documentation and examples}

\begin{figure*}
\includegraphics[width=\textwidth, page=1]{fig/UML-crop}
\caption{\label{fig:class1}Class diagram of Berreman4x4: structure, materials, and module functions. }
\end{figure*}

\begin{figure*}
\includegraphics[width=\textwidth, page=2]{fig/UML-crop}
\caption{\label{fig:class2}Class diagram of Berreman4x4: layers, inhomogeneous materials, and half-spaces.}
\end{figure*}

The source code contains detailed information on the classes and functions, incorporated as docstrings.
They can be conveniently displayed when working with a shell like IPython \cite{IPython}.

Commented examples are bundled with the program and can be run directly from the command line.
They range from simple situations to more complex structures: reflexion on an interface, interferences in a thin layer, reflexion on a Bragg mirror, 90$^\circ$ twisted nematic liquid crystal.
Whenever possible, result from Berreman4x4 is compared with exact analytical result, and show excellent match.
The code is also validated against results presentend in reference~\onlinecite{Fujiwara}.

The UML class diagram of Berreman4x4 is presented on figures~\ref{fig:class1} and \ref{fig:class2}.
A \texttt{Structure} is built from a list of \texttt{Layer} objects, between one front and one back \texttt{HalfSpace} objects.
\texttt{Material} objects are created for defining \texttt{MaterialLayer} objects.
Several classes of non-dispersive materials are provided, but dispersive materials may also be created, like the class \texttt{IsotropicDispersive}.
Additionnal classes of materials can be created by deriving classes \texttt{Material} and \texttt{DispersionLaw}.
Layers with a continuous spatial variation of the permittivity tensor are described with an \texttt{InhomogeneousMaterial} and form an \texttt{InhomogeneousLayer} object.

\section{TODO}

General:
\begin{itemize}
\item Check Berreman's equations again. Which are the assumptions?
\item Verify efficiency of Z.~Lu's method. It seems to be worse than the midpoint method for the twisted nematic example. Strange.
\end{itemize}
%
Source code:
\begin{itemize}
\item Homogeneous layer: implement exact solution with Cayley-Hamilton or Lagrange-Sylvester
\end{itemize}



% Encoding: utf-8

\chapterdoc{Validation examples taken from Fujiwara's book}
\chapterauthor{Olivier Castany}

We reproduce the situation presented by Fujiwara in his book \emph{Spectroscopic Ellipsometry} \cite{Fujiwara}, section~6.4.2, p.~241--243 and section~6.4.1, p.~237--239.
He presents detailed calculations and intermediate steps that are useful for testing our code.


\section{Example of section 6.4.2}

The situation is depicted on figure~\ref{fig:situation642}.
A uniaxial film is formed on a silicon substrate.
The incident medium is air and the silicon substrate has refractive index $n_t = 3.898 + 0.016i$.
The thickness of the film is $d = 100$~nm and the refractive indices are $n_o = 2.0$ and $n_e=2.5$. 
The orientation of the film is given by Euler angles $\phi_E=\pi/4$ and $\theta_E = \pi/4$.
Angle $\phi_E$ is the first Euler angle, inducing a rotation of axis $x$ around axis $z$, leading to axis $x'$.
Angle $\theta_E$ is the seconde Euler angle, inducing a rotation of the axis of the material around $x'$.

\begin{figure}[b]
\includegraphics[width=\columnwidth]{fig/structure-Fujiwara-642}
\caption{\label{fig:situation642}Geometry of the situation treated by Fujiwara in his book, section~6.4.2.}
\end{figure}

In \verb/validation-Fujiwara-642.py/, we reproduce Fujiwara's results with our code and obtain the same values, except for a few places where a sign is reversed.
The reason is that Fujiwara uses the convention of ellipsometry for orienting $E_{rp}$, which is the opposite of our convention.
We also reproduce figure~6.19, p.~242, when the orientation of the anisotropic film is varied.

% Ugly trick to improve column placement...
\vspace{10cm}
\mbox{}
\vspace{7cm}

\section{Example of section 6.4.1}

\begin{figure}[b]
\includegraphics[width=\columnwidth]{fig/structure-Fujiwara-641}
\caption{\label{fig:situation641}Geometry of the situation treated by Fujiwara in his book, section~6.4.1.}
\end{figure}

In \verb/validation-Fujiwara-641.py/, light is reflected in the air by an anisotropic substrate as depicted on figure~\ref{fig:situation641}.
We reproduce figure~6.16, p.~238, when the angle of incidence $\phi_i$ is varied and figure~6.17, p.~239, when the Euler angle $\phi_E$ of the anisotropic substrate is varied.
The difference with section~6.4.2 is that the substrate is anisotropic and the code uses a general \verb/HalfSpace/ instead of an \verb/IsotropicHalfSpace/.




% Encoding: utf-8

\chapterdoc{Example of frustrated total internal reflection}
\chapterauthor{ Olivier Castany, Céline Molinaro}


We reproduced the situation of frustrated total internal reflection. A general and theoretical approach is presented first with simple cases. Then an application with the frustrated internal reflection is presented.


\section{Presentation}

As described on figure~\ref{fig:FTIR}, we consider three dielectrics : front (refractive index $n_f$), air (refractive index $n$ and thickness $d$) and back (refractive index $n_b$).

Without loss of generality, we consider plane waves in the $(x,z)$ direction, i.e. $k_y=0$. The front half-space is isotropic and a plane wave can be decomposed into $s$ and $p$ polarizations.
The $s$ polarization is a wave with perpendicular electric field (\emph{senkrecht}) to the plane of incidence, i.e. along $y$.
The $p$ polarization is a wave with parallel electric field to the plane of incidence.
A plane wave $i$ is incident from the front half-space with incidence angle $\phi_i$ and reflected into a plane wave $r$ with the same angle. Angles are oriented by the $y$ direction. 
A total internal reflection occurs if the angle of incidence is larger than the critical angle
$\Phi_i > \Phi_{ic}$ with $\Phi_{ic}=\arcsin\displaystyle\frac{n}{n_f}$

An evanescent wave is obtained in the adjacent medium. The third medium allows the frustrated total internal reflection. A transmitted plane wave $t$ occurs with an angle $\phi_t$.

\begin{figure}[!h]
\includegraphics[width=\columnwidth]{fig/FTIR}
\caption{\label{fig:FTIR}Frustrated total internal reflection with input and output plane waves. }
\end{figure}

\section{Poynting vector in general cases}
This section contains a description of three general cases whose can be applied to our system. In each description, we are interested in the Poynting vector, which is defined by:
$$\langle \vec{\Pi} \rangle _t = \begin{pmatrix}
\langle \Pi _x \rangle _t\\
\langle \Pi _z \rangle _t
\end{pmatrix}
=\frac{\Re(\vec{E}\times \vec{H}^*)}{2} $$
In the case of a polarisation s, we get $\vec{H}$ from $\vec{E}$ from using a Maxwell equation
$\vec{\nabla}\vec{E} = -\mu_0 \frac{\partial\vec{H}}{\partial t}$.\\
In the case of a polarisation p, we get $\vec{E}$ from $\vec{H}$ from using a Maxwell equation $\vec{\nabla}\vec{H}=\epsilon_0\frac{\partial\vec{H}}{\partial t}$.\\


Calculations have been done for polarisations s and p. We get the same interpretation of the results for the Poynting vector for both polarisations.

\subsection{Plane wave}
In this case, we consider one wave in its medium (like the wave in the back space for the frustrated total internal reflection).\\
\begin{figure}[h!]
\includegraphics[width=\columnwidth]{fig/Plane_wave_real_complex}
\caption{\label{fig:Plane_wave_real_complex}Plane  wave representation when $k_z$ is real (a) and when $k_z$ is an imaginary number (b)}
\end{figure}

On the figure ~\ref{fig:Plane_wave_real_complex}, the wave vectors are represented. For $(a)$ the Poynting vector is parallel to the wave vector. For $(b)$ the Poynting vector is parallel to the real part of the wave vector.\\
\subsubsection{Polarisation s}
The electric field is $\vec{E}=E_0e^{-i\omega t+i(k_zz+k_xx)}\vec{y}$ where $k_x\in \mathbb{R}$ and $k_z\in \mathbb{C}$.
The magnetic field is\\
$$
\vec{H}^*=\frac{E_0^*}{\omega \mu _0}(-k_z^*e^{i\omega t-i(k_z^*z+k_xx)}\vec{x}+k_xe^{+i\omega t-i(k_z^*z+k_xx)}\vec{z}) 
$$ with $k_z=k_z'+ik_z''$.\\
The Poynting vector is:
\begin{equation}
\langle \vec{\Pi} \rangle _t=\displaystyle\frac{\displaystyle\mid E_0\mid ^2e^{-2k_z''z}}{\displaystyle2\omega \mu _0}
\begin{pmatrix}
k_x\\
k_z'
\end{pmatrix}
\end{equation}

\subsubsection{Polarisation p}
The magnetic field is:\\
$$\vec{H}=H_0e^{-i\omega t+i(k_zz+k_xx)}\vec{y}$$\\
The electric field is:\\
$$\vec{E}=\frac{H_0}{\omega\epsilon_0}(k_ze^{-i\omega t+i(k_zz+k_xx)}\vec{x}-k_xe^{-i\omega t+i(k_zz+k_xx)}\vec{z})$$\\
The Poynting vector is:
\begin{equation}
\langle \vec{\Pi} \rangle _t=\displaystyle\frac{\displaystyle\mid H_0\mid ^2e^{-2k_z''z}}{\displaystyle2\omega \epsilon _0}
\begin{pmatrix}
k_x\\
k_z'
\end{pmatrix}
\end{equation}
\subsubsection{Physical interpretation}
In both cases of polarisation, if $k_z\in \mathbb{R}$, the Poynting vecteur is oriented to the wave vector and its power is constant. If $k_z\in i\mathbb{R}$, the Poynting vector is oriented to $\vec{k}_x$ and its power decreases exponentialy along $z$.\\


\subsection{Plane wave reflected by an arbitrary medium}
In this case there are two waves, the incident and the reflected.
\subsubsection{Polarisation s}
$\vec{E}^+$ is the incident wave and $\vec{E}^-$ is the refracted wave. The electric field is
\begin{equation*}
\vec{E}=\vec{E}^++\vec{E}^-
\end{equation*}
The relation between these two components is
\begin{equation*}
E^-_0=r_sE^+_0
\end{equation*}
where $r_s=\Re(r_s)+i\Im(r_s)$ is the amplitude coefficient reflection.\\
The electric field is given by $$\vec{E}=E^+_0(e^{-i\omega t+i(k_xx+k_zz)}+r_se^{-i\omega t+i(k_xx-k_zz)})\vec{y}$$
where $k_x\in \mathbb{R}$ and $k_z\in \mathbb{C}$.\\
The magnetic field is\\
\begin{align*}
\vec{H}^*=\frac{E^{+*}_0}{\omega \mu _0}(&k_z^*(-e^{i\omega t-i(k_xx+k_z^*z)}+r_s^*e^{i\omega t+i(k_zz-k_xx)})\vec{x}\\
&+k_x(e^{i\omega t-i(k_xx+k_z^*z)}+r_s^*e^{i\omega t+i(k_zz-k_xx)})\vec{z}
\end{align*}\\
The Poynting vector is:\\
\begin{align*}
\langle \vec{\Pi} \rangle _t= \displaystyle\frac{\displaystyle \mid E^+_0\mid ^2}{\displaystyle 2\omega \mu _0}\Big\{&
e^{-2k_z''z}
\begin{pmatrix}
k_x\\
k_z'
\end{pmatrix}
+\mid r_s\mid ^2e^{2k_z''z}
\begin{pmatrix}
k_x\\
-k_z'
\end{pmatrix}\\
&+2
\begin{pmatrix}
k_x\Re(r_s)\cos(2k_z'z)\\
k_z''\Im(r_s)\sin(2k_z'z)
\end{pmatrix}
\Big\}
\end{align*}

\subsubsection{Polarisation p}
$\vec{H}^+$ is the incident wave and $\vec{H}^-$ is the refracted wave. The magnetic field is
\begin{equation*}
\vec{H}=\vec{H}^++\vec{H}^-
\end{equation*}
The relation between these two components is
\begin{equation*}
H^-_0=r_pH^+_0
\end{equation*}
where $r_p=\Re(r_p)+i\Im(r_p)$ is the amplitude coefficient reflection.\\
The electric field is given by $$\vec{H}=H^+_0(e^{-i\omega t+i(k_xx+k_zz)}+r_pe^{-i\omega t+i(k_xx-k_zz)})\vec{y}$$
where $k_x\in \mathbb{R}$ and $k_z\in \mathbb{C}$.\\
The electric field is\\
\begin{align*}
\vec{E}=\frac{H^{+}_0}{\omega \epsilon _0}(&k_z^*(e^{-i\omega t+i(k_xx+k_zz)}-r_pe^{-i\omega t+i(k_xx-k_zz)})\vec{x}\\
&-k_x(e^{-i\omega t+i(k_xx+k_zz)}+r_pe^{-i\omega t+i(k_xx-k_zz)})\vec{z}
\end{align*}
The Poynting vector is:\\
\begin{align*}
\langle \vec{\Pi} \rangle _t= \displaystyle\frac{\displaystyle \mid H_0\mid ^2}{\displaystyle 2\omega \epsilon _0}\Big\{&
e^{-2k_z''z}
\begin{pmatrix}
k_x\\
k_z'
\end{pmatrix}
+\mid r_p\mid ^2e^{2k_z''z}
\begin{pmatrix}
k_x\\
-k_z'
\end{pmatrix}\\
&+2
\begin{pmatrix}
k_x\Re(r_p)\cos(2k_z'z)\\
k_z''\Im(r_p)\sin(2k_z'z)
\end{pmatrix}
\Big\}
\end{align*}
\subsubsection{Physical interpretation}
In both cases of polarisation, if $k_z\in \mathbb{R}$, the Poynting vector is constant along $z$. There are interferences along $x$ given by $k_x\cos(\theta _{r'}-2k_z'z)$.
If $k_z\in i\mathbb{R}$, there are interferences along both $x$ and $z$ directions given by $k_x\Re(r_{s,p})\cos(2k_z'z)$ and $k_z''\Im(r_{s,p})\sin(2k_z'z)$ respectively. Along $z$ there is only an interference term.
\begin{figure}[!h]
\includegraphics[width=\columnwidth]{fig/Wave_evanescent_poynting}
\caption{\label{fig:Wave_evanescent_poynting}Poynting vector for evanescents wave 
 and for their superimpose}
\end{figure}
This case is represented on figure ~\ref{fig:Wave_evanescent_poynting}. Each wave has a Poynting vector along $x$. When they are superimpose, interferences occurs on $z$, so the Poynting vector has a component along $z$ too. The Poynting vector addition is not linear.

\subsection{Plane wave reflected by a diopter}
An incident wave and its reflective plane are in a first medium $1$. The waves can be plane or evanescent. In the medium $2$ there is a refracted or transmitted plane wave.  
\subsubsection{Polarisation s}
The amplitude coefficient reflection for the electric field is known:
$$r_s = \frac{k_{z1}-k_{z2}}{k_{z1}+k_{z2}} \ \ \text{and} \ \ t_s=1+r_s$$
The calculations are done in the most general case so $k_{z1}\in \mathbb{C}$ and $k_{z2}\in \mathbb{C}$.\\
$k_{z1}=k_{z1}'+ik_{z1}''$ and $k_{z2}=k_{z2}'+ik_{z2}''$
\begin{equation*}
r_s=\frac{\mid k_{z1}\mid ^2-\mid k_{z2}\mid ^2+i2\Im(k_{z1}k_{z2}^*)}{\mid k_{z1}+k_{z2}\mid^2}
\end{equation*}
The interesting part of $r_s$ is its imaginary part:\\
\begin{equation*}
\Im(r_s) = \frac{2(k_{z2}'k_{z1}''-k_{z1}'k_{z2}'')}{\mid k_{z1}+k_{z2}\mid^2}
\end{equation*}
\subsubsection{Polarisation p}
The amplitude coefficient reflection for the magnetic field is known:
$$
r_p=\frac{\epsilon_2k_{z1}-\epsilon_1k_{z2}}{\epsilon_2k_{z1}+\epsilon_1k_{z2}}
 \ \ \text{and} \ \ t_p = \frac{\cos(\Phi_1)(1-r_p)}{\cos(\Phi_2)}$$
with $\epsilon_i=n_i^2$
The interesting part of $r_p$ is its imaginary part:\\
\begin{equation*}
\Im(r_p) = \frac{2\epsilon_1\epsilon_2(k_{z2}'k_{z1}''-k_{z1}'k_{z2}'')}{\mid k_{z1}+k_{z2}\mid^2}
\end{equation*}
This part is used to know the sign of the third term in the Poynting vector expression.
\subsubsection{Physical interpretation}
The waves can be plane $k_{z}\in \mathbb{R}$ or evanescent $k_{z}\in i\mathbb{R}$.


\section{Application to the frustrated internal total reflection}

\subsection{Electric fields and wave vectors}
To study the frustrated total internal reflection, calculations are done in the case where the angle of incidence is larger than the critical angle :
$\Phi_i > \Phi_{ic}$
Five waves are studied. Each wave has the same wave vector $k_x$.
\begin{center}
$k_x=n_fk_0\sin(\Phi_i)$ and $k_x=n_bk_0\sin(\Phi_t)$\\
\end{center}

The first two are in the front medium where the wave vector along $z$, $k_{fz}\in \mathbb{R}$.
\begin{itemize}
\item  Incident wave  $\vec{E}_i(x,z)=\vec{E}_i(z)e^{-i\omega t+i(k_{fz}z+k_xx)}$
\item Reflected wave  $\vec{E}_r(x,z)=\vec{E}_r(z)e^{-i\omega t+i(-k_{fz}z+k_xx)}$\\
where $k_{fz}=n_fk_0\cos(\Phi _i)$.
\end{itemize}
Two others are in the air where $k_{nz}\in i\mathbb{R}$. One is the transmitted wave from the front medium and the other is the reflected wave.
\begin{itemize}
\item Incident wave 
$\vec{E}_{ni}(x,z)=\vec{E}_ni(z)e^{-i\omega t+i(k_{nz}z+k_xx)}$
\item Reflected wave 
$\vec{E}_{nr}(x,z)=\vec{E}_nr(z)e^{-i\omega t+i(-k_{nz}z+k_xx)}$\\
where $k_{nz}=ik_0\sqrt{n_f^2\sin^2(\Phi _i)-n^2}$.
\end{itemize}
The last wave is in the back medium where $k_{b}\in \mathbb{R}$.
\begin{itemize}
\item Transmitted wave
$\vec{E}_t(x,z)=\vec{E}_t(z)e^{-i\omega t+i(k_{bz}z+k_xx)}$\\
where $k_{bz}=n_bk_0\cos(\Phi _t)$.
\end{itemize}
In this medium the plane wave has an angle \\
$$
\Phi_t=\arcsin (\frac{n_{fz}\sin(\Phi_i)}{n_b})
$$

\subsection{Continuity equations} 
The electric waves follow the $y$ axe.\\
Continuity equations $z=0$
\begin{equation}
\left\lbrace
\begin{array}{ccc}\label{eq:continuity0}
E_{ni}&=r_{fn}E_{nr}+t_{nf}E_i\\
E_r&=r_{nf}E_{i}+t_{fn}E_{nr}
\end{array}\right.
\end{equation}
Continuity equations $z=z_b$
\begin{equation}
\left\lbrace
\begin{array}{ccc}\label{eq:continuityd}
E_t=t_{bn}E_{ni}\\
E_{nr}=r_{bn}E_{ni}
\end{array}\right.
\end{equation}
$E_{ni}$ and $E_{nr}$ can be written at $z=0$ or $z=z_b$. The relations between them are
\begin{equation}
\left\lbrace
\begin{array}{ccc}\label{eq:0tod}
E_{ni}(0)=E_{ni}(d)e^{-ik_{nz}d}\\
E_{nr}(0)=E_{nr}(d)e^{ik_{nz}d}
\end{array}\right.
\end{equation}

With the equations~\eqref{eq:continuity0},~\eqref{eq:continuityd} and~\eqref{eq:0tod}, we got\\
\begin{equation}
E_{ni}(0)=\frac{t_{nf}}{1+r_{nf}r_{bn}e^{i2k_{nz}d}}E_i(0)
\end{equation}

\begin{equation}
E_{nr}(0)=\frac{t_{nf}t_{bn}e^{i2k_{nz}d}}{1+r_{nf}r_{bn}e^{i2k_{nz}d}}E_i(0)
\end{equation}

\begin{equation}\label{eq:reflectivewave}
E_{r}(0)=\frac{r_{fn}+r_{bn}e^{i2k_{nz}d}}{1+r_{nf}r_{bn}e^{i2k_{nz}d}}E_i(0)
\end{equation}

\begin{equation}\label{eq:transmittedwave}
E_{t}(z_b)=\frac{t_{nf}t_{bn}e^{ik_{nz}d}}{1+r_{nf}r_{bn}e^{i2k_{nz}d}}E_i(0)
\end{equation}
According to equation~\eqref{eq:reflectivewave}, the amplitude coefficient reflection for the interface is\\
$$
r=\frac{r_{fn}+r_{bn}e^{i2k_{nz}d}}{1+r_{nf}r_{bn}e^{i2k_{nz}d}}
$$
The reflection coefficient is given by
\begin{equation*}
R=\mid r\mid^2
\end{equation*}
According to equation~\eqref{eq:transmittedwave}, the amplitude coefficient transmission for the interface is\\
$$
t=\frac{t_{nf}t_{bn}e^{ik_{nz}d}}{1+r_{nf}r_{bn}e^{i2k_{nz}d}}
$$
The transmission coefficient is given by
\begin{equation*}
T=\frac{n_b\cos\Phi _t}{n_f\cos\Phi _i}\mid t\mid^2
\end{equation*}
The conservation of energy $R+T=1$ is respected in the cases of polarisation p and polarisation s.


\subsection{Front medium}
In this medium, there are two waves, the incident and the reflected. The total electric field is:
$\vec{E}_f = \vec{E}_i+\vec{E}_r$\\
On the $y$ axe:
\begin{eqnarray*}
E_f = E_i(0)\ 
\big\{&  e^{-i\omega t+i(k_{fz}z+k_xx)} + \cdots  \\
 &+ r_{tot} \ e^{-i\omega t+i(-k_{fz}z+k_xx)} \big\}
\end{eqnarray*}
with 
\begin{eqnarray*}
r_{tot} & = & \frac{r_{fn}+r_{bn}e^{i2k_{nz}d}}{1+r_{nf}r_{bn}e^{i2k_{nz}d}}\\
& = & \Re(r_{tot})+i\Im(r_{tot})
\end{eqnarray*}
%The conugate magnetic field is:\\
%\begin{eqnarray*}
%\vec{H}^*=\frac{E_i(0)^*}{\omega \mu_0}\{&k_{fz}(-e^{i\omega t-i(k_{fz}z+k_xx)}\\
%+r_{tot}^*e^{i\omega t+i(k_{fz}z-k_xx)})\vec{x}\\
%+&k_x(e^{i\omega t-i(k_{fz}z+k_xx)}\\
%+r_{tot}^*e^{i\omega t+i(k_{fz}z-k_xx)})\vec{z}\}
%\end{eqnarray*}
The Poynting vector:\\
\begin{align*}
\langle \vec{\Pi} \rangle _t =\displaystyle\frac{\displaystyle\mid E_i(0)\mid ^2}{\displaystyle2\omega \mu_0}\Big\{&
\begin{pmatrix}
k_x\\
k_{fz}
\end{pmatrix}
+\mid r_{tot}\mid^2
\begin{pmatrix}
k_x\\
-k_{fz}
\end{pmatrix}\\
&+2
\begin{pmatrix}
\Re(r_{tot})\cos(2k_{fz}z\\
0
\end{pmatrix} 
\Big\}
\end{align*}

\subsection{Air medium}
In this medium, there are two waves. The total electric field is:
$\vec{E}_n=\vec{E}_{ni}+\vec{E}_{nr}$\\
On the $y$ axe:\\
\begin{eqnarray*}
E_n = E_{ni}(0)\ 
\big\{&  e^{-i\omega t+i(k_{nz}z+k_xx)} + \cdots  \\
 &+ r_n \ e^{-i\omega t+i(-k_{nz}z+k_xx)} \big\}
\end{eqnarray*}
with:
\begin{eqnarray*}
r_n & = & r_{bn}e^{i2k_{nz}d}\\
& = & \Re(r_n)+i\Im(r_n)
\end{eqnarray*}

%The conjugate magnetic fiels has been calculated in the most general case, with $k_{nz}\in \mathbb{C}$:\\
%\begin{eqnarray*}
%\vec{H}^*=\frac{E_{ni}(0)^*}{\omega \mu_0}\{
%&k_{nz}^*\big(-e^{i\omega t-i(k_{nz}^*z+k_xx)}\\
%+r_n^*e^{i\omega t+i(k_{nz}^*z-k_xx)}\big)\vec{x}\\
%+&k_x\big(e^{i\omega t-i(k_{nz}^*z+k_xx)}\\
%+e^{i\omega t+i(k_{nz}^*z-k_xx)}\big)\vec{z}\}
%\end{eqnarray*}
%We write:\\
%$$
%k_{nz}=k_{nz}'+ik_{nz}''
%$$
%The Poynting vector is:\\
%$
%\langle \vec{\Pi} \rangle _t =\displaystyle\frac{\displaystyle\mid E_{ni}(0)\mid ^2}{\displaystyle2\omega \mu_0}\Big\{
%e^{-2k_{nz}''z}
%\begin{pmatrix}
%k_x\\
%k_{nz}'
%\end{pmatrix}\\
%+\mid r_n\mid ^2e^{2k_{nz}''z}
%\begin{pmatrix}
%k_x\\
%-k_{nz}'
%\end{pmatrix}\\
%+2\mid r_n\mid
%\begin{pmatrix}
%k_x\cos(\theta _n-2k_{nz}'z)\\
%k_{nz}''\sin(\theta _n -2k_{nz}'z)
%\end{pmatrix}
%\Big\}
%$\\
In the case of the frustrated total internal reflexion, the wave in the air medium is an evanescent wave, $k_{nz} \in i\mathbb{R}$. The Poynting vector is:\\
$
\langle \vec{\Pi} \rangle _t =\displaystyle\frac{\displaystyle\mid E_{ni}(0)\mid ^2}{\displaystyle2\omega \mu_0}\Big\{
e^{2k_{nz}z}
\begin{pmatrix}
k_x\\
0
\end{pmatrix}
+\mid r_n\mid ^2e^{-2k_{nz}z}
\begin{pmatrix}
k_x\\
0
\end{pmatrix}\\
+2
\begin{pmatrix}
k_x\Re(r_n)\\
ik_{nz}^*\Im(r_n)
\end{pmatrix}
\Big\}
$\\
The third term is an interference terms. Along $z$ we get:\\
$$
ik_{nz}^*\sin(\theta _n) =\frac{ k_0^3n_b\cos(\Phi _t)(n_f^2\sin ^2(\Phi _i)-n^2)e^{ik_{nz}d}}{\mid k_{nz}+k_{fz}\mid ^2}
$$
This expression is positive so the interference term is oriented to the positive $z$.
\subsection{Back medium}
In this medium, there is one wave, the transmitted wave.With equation~\eqref{eq:transmittedwave}, on the $y$ axe:\\
\begin{equation*}
E_t = E_i(0)te^{-i\omega t+i(k_bz+k_xx)}
\end{equation*}
with:
$$
t = \frac{t_{nf}t_{bn}e^{ik_{nz}d}}{1+r_{nf}r_{bn}e^{i2k_{nz}d}}
$$
%The conjugate magnetic field is:\\
%\begin{eqnarray*}
%\vec{H}^*= \frac{E_i(0)^*}{\omega \mu_0}\Big\{-k_t^*be^{i\omega t-i(k_bz+k_xx)}\vec{x}\\
%+k_xt^*e^{i\omega t-i(k_bz+k_xx)}\vec{z}\Big\}
%\end{eqnarray*}
The Poynting vector is:\\
$$
\langle \vec{\Pi} \rangle _t = \frac{\mid E_i(0)\mid ^2\mid t\mid ^2}{2\omega \mu_0}
\begin{pmatrix}
k_x\\
k_b
\end{pmatrix}
$$
% Encoding: utf-8

\chapterdoc{Various examples}
\chapterauthor{Olivier Castany, Céline Molinaro}

\section{Reflection on an interface}
This is the  simplest case as shown on figure ~\ref{fig:Reflection-interface}. A wave is reflected and refracted by an interface. This interface is characterized by the  transmission and reflection power coefficient. See source code for further details.
\begin{figure}[!h]
\includegraphics[width=\columnwidth]{fig/Reflection-interface}
\caption{\label{fig:Reflection-interface}Input and output plane waves at the interface.}
\end{figure}


\section{Bragg Mirror}
Figure ~\ref{fig:Bragg} represents a Bragg mirror with two layers. Each layers is composed of two thin dielectric layers : SiO$_2$ and TiO$_2$. An incident wave, its reflecteds waves and its transmitted wave are represented. Bragg mirrors are used to produce ultra-high relfectivity. This example is used to draw the global reflection power coefficient for all layers depending on the wave length. See source code for further details.
\begin{figure}[!h]
\includegraphics[width=\columnwidth]{fig/Bragg}
\caption{\label{fig:Bragg}Bragg Mirror with two layers}
\end{figure}

\section{Cholesteric liquid crystal}
The structure of a cholesteric crystal is helical. The director vector is rotated.
\begin{figure}[!h]
\includegraphics[width=\columnwidth]{fig/cholesteric}
\caption{\label{fig:cholesteric}Director vector of a cholesteric crystal where p is the pitch}
\end{figure}

%%%%%%%%%%%%%%%%%%%%%%%%%%%%%%%%%%%%%%%%%%%%%%%%%%%%%%%%%%%%%%%%%%%%%%%%%%%%%

\bibliographystyle{unsrtnat}
\bibliography{documentation}

\end{document}

